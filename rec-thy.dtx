% \NeedsTeXFormat{LaTeX2e}
% Release Notes see README
\RequirePackage{expl3}
\RequirePackage{ltxcmds}
\RequirePackage{ifxetex,ifpdf}
\RequirePackage{suffix}
\RequirePackage{ifmtarg}
\RequirePackage{xifthen}
\RequirePackage{xkeyval}
\RequirePackage{etoolbox}
\RequirePackage{pict2e,picture}
\RequirePackage{xparse}
\def\recthyVersion{3.7}
\ProvidesExplPackage
  {rec-thy}
  {2021/02/28}
  {\recthyVersion}
  {Provides commands for writing up recursion theory papers}



\ExplSyntaxOff

% \ProvidesExplPackage{rec-thy}{2020/06/04}{3.5}{Provides commands for writing up recursion theory papers} % Description
% \ProvidesPackage{rec-thy}[2021/02/05 v\recthyVersion Provides commands for writing up recursion theory papers]
\makeatletter
% \newcommand*{\@curpack}{\@currname} %rec-thy.sty
\newcommand*{\@packinfo}[1]{\PackageInfo{rec-thy}{#1}}
\newcommand*{\recthy@strip}[1]{%
\ifcsname #1\endcsname%
\else%
\expandafter\edef\csname #1\endcsname{\expandafter\noexpand\csname @recthy@#1\endcsname}
\fi
}
%leaves us a command \arg defined to do \@recthy@arg
% \ProvidesPackage{\@curpack}  %file name is package name



\robustify\(
\robustify\)
\ltx@ifpackageloaded{unicode-math}{}{\RequirePackage{mathrsfs}}%\RequirePackage{mathbbol}}
  \ltx@ifpackageloaded{amsmath}{}{\RequirePackage{amsmath}}
  \ltx@ifpackageloaded{amssymb}{}{\RequirePackage{amssymb}}
  \ltx@ifpackageloaded{amsfonts}{}{\RequirePackage{amsfonts}}
\AtBeginDocument{
% \@ifundefined{hyperref}{\providecommand{\texorpdfstring}[2]{def}  }

  \ltx@ifpackageloaded{mathtools}{}{\RequirePackage{mathtools}}
  
  % Undertilde doesn't exist in texlive so we remove the dependency and define it again.
  \ltx@ifpackageloaded{undertilde}{}{
      \newlength\knuthian@fdfive
      \def\mathpal@save#1{\let\was@math@style=#1\relax}
      \def\utilde#1{\mathpalette\mathpal@save
                    {\setbox124=\hbox{$\was@math@style#1$}%
      \setbox125=\hbox{$\fam=3\global\knuthian@fdfive=\fontdimen5\font$}
      \setbox125=\hbox{$\widetilde{\vrule height 0pt depth 0pt width \wd124}$}%
                     \baselineskip=1pt\relax
                     \vtop{\copy124\copy125\vskip -\knuthian@fdfive}}}
      }
}

% \ifxetex  %workaround for bug...hopefully can dispense on later
\ifxetex
\else
    \def\symbf#1{\mathbf{#1}}
\fi



% \DeclareMathAlphabet{\mathbrush}{T1}{pbsi}{xl}{n}
% \newcommand*{\mathbrush}[1]{\mathsfit{#1}}
% \ifxetex
%   \RequirePackage{xltxtra}
%   \RequirePackage{unicode-math}
%   % \RequirePackage{fontspec,xunicode}
%   % \defaultfontfeatures{Mapping=tex-text,Scale=MatchLowercase} % To support LaTeX quoting style
%   % \input{/Users/TruePath/Library/texmf/tex/headers/appearence.tex}
% \else
%   % \usepackage[utf8]{inputenc}
%   % \usepackage[T1]{fontenc}
%   % \RequirePackage{euscript}
%   % \newcommand*{\matheu}{\EuScript}

%   \ifpdf
%   \RequirePackage{cmap} % you can search in the pdf for umlauts and ligatures
%   \fi
% \fi
% \RequirePackage{stmaryrd}
% \RequirePackage{marvosym}

\def\PMG@arginit{\let\@PMG@parenarg\@PMG@undefined\let\@PMG@braketarg\@PMG@undefined}

\def\@recthy@family{\@currname.\@currext} %The default family for this package...filename.ext
%Define boolean options true if option passed false otherwise


\newif\ifrecthy@nohyphenation       \recthy@nohyphenationfalse
\newif\ifrecthy@mrref           \recthy@mrreffalse
\newif\ifrecthy@nomath          \recthy@nomathfalse
\newif\ifrecthy@nodelim         \recthy@nodelimfalse
\newif\ifrecthy@nonames
\recthy@nonamesfalse
\newif\ifrecthy@nosets
\recthy@nosetsfalse
\newif\ifrecthy@noquants\recthy@noquantsfalse
\newif\ifrecthy@nospaces        \recthy@nospacesfalse
\newif\ifrecthy@nostrings       \recthy@nostringsfalse
\newif\ifrecthy@notrees         \recthy@notreesfalse
\newif\ifrecthy@nopair          \recthy@nopairfalse
\newif\ifrecthy@nonames         \recthy@nonamesfalse
\newif\ifrecthy@nosetrels       \recthy@nosetrelsfalse
\newif\ifrecthy@noordinalnotations  \recthy@noordinalnotationsfalse
\newif\ifrecthy@nosyntax        \recthy@nosyntaxfalse
\newif\ifrecthy@nodegclasses        \recthy@nodegclassesfalse
\newif\ifrecthy@noforcing       \recthy@noforcingfalse
\newif\ifrecthy@nodegrees       \recthy@nodegreesfalse
\newif\ifrecthy@nocomputations      \recthy@nocomputationsfalse
\newif\ifrecthy@nostructresets \recthy@nostructresetsfalse
\newif\ifrecthy@noreqhelper \recthy@noreqhelperfalse
\newif\ifrecthy@nosubfun \recthy@nosubfunfalse
\newif\ifrecthy@nopfcases \@ifclassloaded{beamer}{\recthy@nopfcasestrue}{\recthy@nopfcasesfalse}
\newif\ifrecthy@beamersupp \@ifclassloaded{beamer}{\recthy@beamersupptrue}{\recthy@beamersuppfalse}
\newif\ifrecthy@noprioritytrees \recthy@noprioritytreesfalse
\newif\ifrecthy@nosuppprioritytrees \recthy@nosuppprioritytreestrue
\newif\ifrecthy@hyperreqs \recthy@hyperreqstrue
\newif\ifrecthy@nooperators \recthy@nooperatorsfalse
\newif\ifrecthy@fix@typeset \recthy@fix@typesettrue
\newif\ifrecthy@nodoubleangles \recthy@nodoubleanglesfalse
\newif\ifrecthy@stepsenv \recthy@stepsenvfalse
\newif\ifrecthy@compatThreeSixOrEarlier \recthy@compatThreeSixOrEarlierfalse

\DeclareOptionX{nohyphenation}{\recthy@nohyphenationtrue}
\DeclareOptionX{mrref}{\recthy@mrreftrue}
\DeclareOptionX{nomath}{\recthy@nomathtrue}
\DeclareOptionX{nodelim}{\recthy@nodelimtrue}
\DeclareOptionX{nonames}{\recthy@nonamestrue}
\DeclareOptionX{nosets}{\recthy@nosetstrue}
\DeclareOptionX{noquants}{\recthy@noquantstrue}
\DeclareOptionX{nospaces}{\recthy@nospacestrue}
\DeclareOptionX{nostrings}{\recthy@nostringstrue}
\DeclareOptionX{notrees}{\recthy@notreestrue}
\DeclareOptionX{nopair}{\recthy@nopairtrue}
\DeclareOptionX{nosetrels}{\recthy@nosetrelstrue}
\DeclareOptionX{noordinalnotations}{\recthy@noordinalnotationstrue}
\DeclareOptionX{nosyntax}{\recthy@nosyntaxtrue}
\DeclareOptionX{nodegclasses}{\recthy@nodegclassestrue}
\DeclareOptionX{noforcing}{\recthy@noforcingtrue}
\DeclareOptionX{nodegrees}{\recthy@nodegreestrue}
\DeclareOptionX{nocomputations}{\recthy@nocomputationstrue}
\DeclareOptionX{nostructresets}{\recthy@nostructresetstrue}
\DeclareOptionX{noreqhelper}{\recthy@noreqhelpertrue}
\DeclareOptionX{nosubfun}{\recthy@nosubfuntrue}
\DeclareOptionX{nopfcases}{\recthy@nopfcasestrue}
\DeclareOptionX{noprioritytrees}{\recthy@noprioritytreestrue}
\DeclareOptionX{nohyperreqs}{\recthy@hyperreqsfalse}
\DeclareOptionX{nooperators}{\recthy@nooperatorstrue}
\DeclareOptionX{nofixtype}{\recthy@fix@typesetfalse}
\DeclareOptionX{nodoubleangles}{\recthy@nodoubleanglestrue}
\DeclareOptionX{steps}{\recthy@stepsenvtrue}

\DeclareOptionX{compat31}{\recthy@compat@threeOne}
\DeclareOptionX{compat35}{\recthy@compat@threeFive}
\DeclareOptionX{compat36}{\recthy@compat@threeSix}

\DeclareOptionX{suppPriorityTrees}{
\ifrecthy@compatThreeSixOrEarlier
   \@packinfo{Enabling depreciated Support Priority Tree functionality.  These functions are only accessible in compatibility mode and should not be used in new documents}
   \recthy@nosuppprioritytreesfalse
\else
  \PackageWarningNoLine{rec-thy}{Suplemental Priority Tree option depreciated.  Combine with one of the options compat36,compat32 or compat 31 to access these functions.}
\fi
}

\newcommand*{\recthy@rollback@threeTwo}{}
\newcommand*{\recthy@rollback@threeSix}{}
\newcommand*{\recthy@rollback@threeSeven}{}
\newcommand*{\recthy@compat@threeSix}{\recthy@compatThreeSixOrEarliertrue\AtEndOfPackage{\recthy@rollback@threeSeven}}
\newcommand*{\recthy@compat@threeFive}{\recthy@compatThreeSixOrEarliertrue\recthy@compat@threeSix\AtEndOfPackage{\recthy@rollback@threeSix}}
\newcommand*{\recthy@compat@threeOne}{\recthy@compatThreeSixOrEarliertrue\recthy@compat@threeFive\AtEndOfPackage{\recthy@rollback@threeTwo}}

%\DeclareOptionX{nomodule}{\recthy@nomoduletrue}

\DeclareOptionX{degjoin}{\def\Tjoin{\Tdegjoin}}

\let\@recthy@modulescr\mathcal
\DeclareOptionX{modulescr}[mathcal]{\letcs{\@recthy@modulescr}{#1}}

\def\@recthy@ballsymb@default{\mathscr{B}}
\let\@recthy@ballsymb\@recthy@ballsymb@default
\DeclareOptionX{ballsymb}[@recthy@ballsymb@default]{\letcs{\@recthy@ballsymb}{#1}}










\let\pmg@casefont\textsc
\DeclareOptionX{pfcasefont}[textsc]{\letcs{\pmg@casefont}{#1}}



\let\@recthy@reqscr\mathscr
\DeclareOptionX{reqscr}[mathscr]{\letcs{\@recthy@reqscr}{#1}}
% \DeclareOptionX{beamerreq}[@recthy@EmptyStr@Default]{\letcs{\@recthy@EmptyStr}{#1}}








% \newcommand*{\@recthy@REA}{\ensuremath{\text{REA}}}
% \WithSuffix\def\@recthy@REA[#1]{#1-\@recthy@REA\relax}
% \newcommand*{\@recthy@CEA}{\ensuremath{\text{CEA}}}
% \WithSuffix\def\@recthy@CEA[#1]{#1-\@recthy@CEA\relax}

\DeclareOptionX{reterms}{%
    \let\@recthy@ce=\@recthy@re
    \let\@recthy@computable=\@recthy@recursive
    \let\@recthy@computablyEnumerable=\@recthy@recursivelyEnumerable
    \let\@recthy@ComputablyEnumerable=\@recthy@RecursivelyEnumerable
    } %Make all references to ce change to re

    \DeclareOptionX{reaterms}{%
        \let\@recthy@CEA=\@recthy@REA
        }

\DeclareOptionX{ceterms}{%
    \let\@recthy@re=\@recthy@ce
    \let\@recthy@recursive=\@recthy@computable
    \let\@recthy@recursivelyEnumerable=\@recthy@computablyEnumerable
    \let\@recthy@RecursivelyEnumerable=\@recthy@ComputablyEnumerable
    } %Make all references to re change to ce

\DeclareOptionX{ceaterms}{%
            \let\@recthy@REA=\@recthy@CEA
            }

\def\@recthy@setdiffSYMdefault{-}
\let\@recthy@setdiffSYM=\@recthy@setdiffSYMdefault
\DeclareOptionX{diffsym}[@recthy@setdiffSYMdefault]{\letcs{\@recthy@setdiffSYM}{#1}}
% \newcommand*{\@recthy@Cform@default}{\prescript{\mathcal{C}}{}{\Pi}}


\newcommand*{\@recthy@CDelta@default}{\prescript{\mathcal{C}}{}{\Delta}}
\newcommand*{\@recthy@CSigma@default}{\prescript{\mathcal{C}}{}{\Sigma}}
\newcommand*{\@recthy@CPi@default}{\prescript{\mathcal{C}}{}{\Pi}}
\let\@recthy@CDelta=\@recthy@CDelta@default
\let\@recthy@CSigma=\@recthy@CSigma@default
\let\@recthy@CPi=\@recthy@CPi@default
\DeclareOptionX{cdeltasym}[@recthy@CDelta@default]{\letcs{\@recthy@CDelta}{#1}}
\DeclareOptionX{csigmasym}[@recthy@CSigma@default]{\letcs{\@recthy@CSigma}{#1}}
\DeclareOptionX{cpisym}[@recthy@CPi@default]{\letcs{\@recthy@CPi}{#1}}
% \DeclareOptionX{cformsym}[@recthy@Cform@default]{\letcs{\@recthy@CPi}{#1}\letcs{\@recthy@CSigma}{#1}\letcs{\@recthy@CDelta}{#1}}




\def\@recthy@recfSYM@default{\phi}
\let\@recthy@recfSYM=\@recthy@recfSYM@default
\def\@recthy@recfnlSYM@default{\Phi}
\let\@recthy@recfnlSYM=\@recthy@recfnlSYM@default
\def\@recthy@useSYM@default{\ltx@ifpackageloaded{unicode-math}{\symbffrak{u}}{\mathfrak{u}}}
\let\@recthy@useSYM=\@recthy@useSYM@default
\DeclareOptionX{recfnlsym}[@recthy@recfnlSYM@default]{\letcs{\@recthy@recfnlSYM}{#1}}
\DeclareOptionX{recfsym}[@recthy@recfSYM@default]{\letcs{\@recthy@recfSYM}{#1}}
\DeclareOptionX{usesym}[@recthy@useSYM@default]{\letcs{\@recthy@useSYM}{#1}}

\let\@recthy@EmptyStr=\@recthy@EmptyStr@Default
\let\@recthy@concatSYM=\@recthy@concatSYM@Default
\def\@recthy@langle{\langle}
\def\@recthy@rangle{\rangle}
\let\@recthy@lstrdelim=\@recthy@langle
\let\@recthy@rstrdelim=\@recthy@rangle
\let\@recthy@lcodedelim=\@recthy@langle
\let\@recthy@rcodedelim=\@recthy@rangle
\let\@recthy@lpairdelim=\@recthy@langle
\let\@recthy@rpairdelim=\@recthy@rangle
\def\@recthy@pairsup@default{2}
\let\@recthy@pairsup=\@recthy@pairsup@default

\def\@recthy@concatSYM@Default{\widehat{\phantom{X}}}
\def\@recthy@EmptyStr@Default{\mathopen{\langle}\mathclose{\rangle}}
\let\@recthy@concatSYM\@recthy@concatSYM@Default
\let\@recthy@EmptyStr\@recthy@EmptyStr@Default

\DeclareOptionX{emptystr}[@recthy@EmptyStr@Default]{\letcs{\@recthy@EmptyStr}{#1}}
\DeclareOptionX{concatsym}[@recthy@concatSYM@Default]{\letcs{\@recthy@concatSYM}{#1}}
\DeclareOptionX{lstrdelim}[@recthy@langle]{\letcs{\@recthy@lstrdelim}{#1}}
\DeclareOptionX{rstrdelim}[@recthy@rangle]{\letcs{\@recthy@rstrdelim}{#1}}
\DeclareOptionX{lcodedelim}[@recthy@langle]{\letcs{\@recthy@lcodedelim}{#1}}
\DeclareOptionX{rcodedelim}[@recthy@rangle]{\letcs{\@recthy@rcodedelim}{#1}}
\DeclareOptionX{lpairdelim}[@recthy@langle]{\letcs{\@recthy@lpairdelim}{#1}}
\DeclareOptionX{rpairdelim}[@recthy@rangle]{\letcs{\@recthy@rpairdelim}{#1}}
\DeclareOptionX{pairsup}[@recthy@pairsup@default]{\letcs{\@recthy@pairsup}{#1}}





% \ProcessOptions*
% \ProcessOptionsX<\@recthy@family>
\ProcessOptionsX

%misc tools

\def\@recthy@ignorespacesandallparsafterend{\ifhmode\unskip\fi
\aftergroup\@recthy@useignorespacesandallpars}
\def\@recthy@useignorespacesandallpars#1\ignorespaces\fi{%
#1\fi\@recthy@ignorespacesandallpars}
\def\@recthy@ignorespacesandallpars{%
  \@ifnextchar\par
    {\expandafter\@recthy@ignorespacesandallpars\@gobble}%
    {}%
}


\ifrecthy@fix@typeset
      %Defining \overbar
      \let\save@mathaccent\mathaccent
  \newcommand*\if@single[3]{%
    \setbox0\hbox{${\mathaccent"0362{#1}}^H$}%
    \setbox2\hbox{${\mathaccent"0362{\kern0pt#1}}^H$}%
    \ifdim\ht0=\ht2 #3\else #2\fi
    }
  %The bar will be moved to the right by a half of \macc@kerna, which is computed by amsmath:
  \newcommand*\rel@kern[1]{\kern#1\dimexpr\macc@kerna}
  %If there's a superscript following the bar, then no negative kern may follow the bar;
  %an additional {} makes sure that the superscript is high enough in this case:
  \providecommand*\overbar{\relax}
  \renewcommand*\overbar[1]{\@ifnextchar^{{\over@bar{#1}{0}}}{\over@bar{#1}{1}}}
  %Use a separate algorithm for single symbols:
  \newcommand*\over@bar[2]{\if@single{#1}{\over@bar@{#1}{#2}{1}}{\over@bar@{#1}{#2}{2}}}
  \newcommand*\over@bar@[3]{%
    \begingroup
    \def\mathaccent##1##2{%
  %Enable nesting of accents:
      \let\mathaccent\save@mathaccent
  %If there's more than a single symbol, use the first character instead (see below):
      \if#32 \let\macc@nucleus\first@char \fi
  %Determine the italic correction:
      \setbox\z@\hbox{$\macc@style{\macc@nucleus}_{}$}%
      \setbox\tw@\hbox{$\macc@style{\macc@nucleus}{}_{}$}%
      \dimen@\wd\tw@
      \advance\dimen@-\wd\z@
  %Now \dimen@ is the italic correction of the symbol.
      \divide\dimen@ 3
      \@tempdima\wd\tw@
      \advance\@tempdima-\scriptspace
  %Now \@tempdima is the width of the symbol.
      \divide\@tempdima 10
      \advance\dimen@-\@tempdima
  %Now \dimen@ = (italic correction / 3) - (Breite / 10)
      \ifdim\dimen@>\z@ \dimen@0pt\fi
  %The bar will be shortened in the case \dimen@<0 !
      \rel@kern{0.6}\kern-\dimen@
      \if#31
        \overline{\rel@kern{-0.6}\kern\dimen@\macc@nucleus\rel@kern{0.4}\kern\dimen@}%
        \advance\dimen@0.4\dimexpr\macc@kerna
  %Place the combined final kern (-\dimen@) if it is >0 or if a superscript follows:
        \let\final@kern#2%
        \ifdim\dimen@<\z@ \let\final@kern1\fi
        \if\final@kern1 \kern-\dimen@\fi
      \else
        \overline{\rel@kern{-0.6}\kern\dimen@#1}%
      \fi
    }%
    \macc@depth\@ne
    \let\math@bgroup\@empty \let\math@egroup\macc@set@skewchar
    \mathsurround\z@ \frozen@everymath{\mathgroup\macc@group\relax}%
    \macc@set@skewchar\relax
    \let\mathaccentV\macc@nested@a
  %The following initialises \macc@kerna and calls \mathaccent:
    \if#31
      \macc@nested@a\relax111{#1}%
    \else
  %If the argument consists of more than one symbol, and if the first token is
  %a letter, use that letter for the computations:
      \def\gobble@till@marker##1\endmarker{}%
      \futurelet\first@char\gobble@till@marker#1\endmarker
      \ifcat\noexpand\first@char A\else
        \def\first@char{}%
      \fi
      \macc@nested@a\relax111{\first@char}%
    \fi
    \endgroup
  }
\else
  \@packinfo{Not loading Fix Typesetting}
\fi


%Strings



  

  \newcommand*{\@recthy@pair}[2]{\mathopen{\@recthy@lpairdelim} #1, #2 \mathclose{\@recthy@rpairdelim}^\@recthy@pairsup}
  \newcommand*{\@recthy@str}[1]{\mathopen{\@recthy@lstrdelim}#1\mathclose{\@recthy@rstrdelim}}
  \newcommand*{\@recthy@code}[1]{\mathopen{\@recthy@lcodedelim}#1\mathclose{\@recthy@rcodedelim}}


  \appto{\recthy@rollback@threeSix}{
    \renewcommand*{\@recthy@pair}[2]{\mathopen{\@recthy@llangle} #1, #2 \mathclose{\@recthy@rrangle}}
    \def\@recthy@EmptyStr@Default{\mathopen{\langle}\mathclose{\rangle}}
    % \renewcommand*{\@recthy@str}[1]{\mathopen{\langle}#1\mathclose{\rangle}}
    \renewcommand*{\@recthy@code}[2][]{\mathopen{\@recthy@llangle}#2\mathclose{\@recthy@rrangle}^{#1}}
    }
\NewDocumentCommand{\incompat@recthy}{o}{\mathrel{\mid}\IfValueTF{#1}{_{#1}}{}}
\NewDocumentCommand{\compat@recthy}{o}{\mathrel{\not\mid}\IfValueTF{#1}{_{#1}}{}}

\ifrecthy@nostrings
    \@packinfo{No strings loading.}
\else
    \@packinfo{Strings loading.}
    \appto{\recthy@rollback@threeSix}{\newcommand*{\closedn}[1]{{#1}^\blacktriangledown}}
    \let\code=\@recthy@code
    \let\str=\@recthy@str
    \newcommand*{\EmptyStr}{\@recthy@EmptyStr}
    \newcommand*{\estr}{\EmptyStr}
    \newcommand*{\decode}[2]{\left(#1\right)_{#2}}
    \let\godelpair=\@recthy@pair
    \let\gpair=\@recthy@pair
    \newcommand*{\concat}{\@recthy@concatSYM}
    \WithSuffix\def\concat[#1]{\concat\str{#1}}
    \newcommand*{\strpred}[1]{{#1}^{-}}
    \providecommand*{\lh}[2][]{\@recthy@abs{#2}_{#1}} %% Gives length of a string

    \providecommand*{\incompat}{\incompat@recthy} %incompatible stringes FIX SPACING
    \providecommand*{\incomp}{\incompat@recthy}
    \providecommand*{\compat}{\compat@recthy}
    \providecommand*{\nincompat}{\compat@recthy}
    \providecommand*{\nincomp}{\compat@recthy}
    % \newcommand*{\setcol}[2]{{#1}^{[#2]}}
    \ExplSyntaxOn %thx stackexghance
      \NewDocumentCommand \setcol {m m}
      {
        \bool_lazy_and:nnTF
          { \int_compare_p:nNn { \tl_count:n {#1} } = 3 }
          { \tl_if_head_eq_meaning_p:nN {#1} \setcol }
          {
            \pgerdes_setcol_special:nnnn #1 {#2}
          }
          {
            \pgerdes_setcol_normal:nn {#1} {#2}
          }
      }
      \cs_new:Npn \pgerdes_setcol_special:nnnn #1 #2 #3 #4
      {
        {#2} \sp { [#3] [#4] }
      }
      \cs_new:Npn \pgerdes_setcol_normal:nn #1 #2
      {
        {#1} \sp { [#2] }
      }
    \ExplSyntaxOff

    \ifrecthy@nopair
        \@packinfo{No redefining pair.}
    \else
        \@packinfo{pair=godelpair.}
        \let\pair=\@recthy@pair
    \fi
\fi

\ifrecthy@noprioritytrees
    \@packinfo{Not loading Priority Tree Helper}
\else
    \@packinfo{Loading Priority Tree Helper.}
    \providecommand{\PriorityTree}{\mathbb{T}}
    \providecommand{\tpath}[1][]{\ifthenelse{\isempty{#1}}{\symbf{f}}{\mathbb{f}_{#1}}}
    \providecommand{\leftof}{\mathbin{<_L}}
    \providecommand{\leftofeq}{\mathbin{\leq_L}}
    \providecommand{\rightof}{\mathbin{>_L}}
    \providecommand{\rightofeq}{\mathbin{\geq_L}}
    \providecommand{\nleftof}{\mathbin{\nless_L}}
    \providecommand{\nleftofeq}{\mathbin{\nleq_L}}
    \providecommand{\nrightof}{\mathbin{\ngtr_L}}
    \providecommand{\nrightofeq}{\mathbin{\ngeq_L}}
    \NewDocumentCommand{\PriorityTreeModule}{mO{}m}{\ensuremath{{\@recthy@modulescr{#1}}^{#2}_{#3} }}
    \let\module=\PriorityTreeModule
    \providecommand{\ball}[3][]{{\@recthy@ballsymb}^{#1}(#2, #3)}
\fi


\ifrecthy@nosuppprioritytrees
\else
    \providecommand*{\Astages}[1]{\mathcal{A}({#1})}
    \providecommand*{\Vstages}[1]{\mathcal{V}(#1)}
    \providecommand{\reqof}[2][]{\mathbb{r}(#2)}
    \let\modof=\reqof
    \providecommand{\ancestor}{\succ}
    \providecommand{\descendant}{\prec}
    \providecommand{\ancestoreq}{\succeq}
    \providecommand{\descendanteq}{\preceq}
    \let\ances=\ancestor
    \let\anceseq=\ancestoreq
    \let\desc=\descendant
    \let\desceq=\descendanteq
\fi


\ifrecthy@nopfcases
    \@packinfo{Not loading Proof Cases Helper}
\else
    \@packinfo{Loading Proof Cases Helper.}
    \RequirePackage{enumitem}
    \appto{\recthy@rollback@threeSeven}{
        \setlist[pfcasesnonum]{
          label={\pmg@casefont{Case}}\protect\thiscase:~,
          align=left,
          labelsep=!,
          itemindent=-1em,
        }
        \setlist[pfcasesnumbered]{
          align=left,
          labelsep=!,
          itemindent=-1em,
        }
        \setlist[pfcasesnumbered,1]{
          label={{\pmg@casefont{Case}}~\arabic*\protect\thiscase:~},
        }
        \setlist[pfcasesnumbered,2]{
          label={{\pmg@casefont{Case}}~\arabic{pfcasesnumberedi}\alph*:\protect\thiscase.~},
        }
        \setlist[pfcasesnumbered,3]{
          label={{\pmg@casefont{Case}}~\arabic{pfcasesnumberedi}\alph{pfcasesnumberedii}.\roman*:\protect\thiscase.~},
        }
        \renewenvironment{pfcases*}{
            \ProvideDocumentCommand{\case}{o}{
                \IfNoValueTF{##1}{
                    \def\thiscase{}%
                }{
                    \def\thiscase{~##1}%
                }
                \item  \ltx@ifpackageloaded{cleveref}{%
                \cref@constructprefix{pfcases}{\cref@result}%
                \protected@xdef\cref@currentlabel{[\pfcasecounter@pmg][][\cref@result]##1}}{}  \protected@edef\@currentlabel{##1}%
        }
        \begin{pfcasesnonum}
        }
        {\end{pfcasesnonum}}
        \renewenvironment{pfcases}{
            \ProvideDocumentCommand{\case}{o}{
                \IfNoValueTF{##1}{
                    \def\thiscase{}%
                }{
                    \def\thiscase{~##1}%
                }
                \item \ltx@ifpackageloaded{cleveref}{%
                \cref@constructprefix{\pfcasecounter@pmg}{\cref@result}
                \protected@xdef\cref@currentlabel{[\pfcasecounter@pmg][][\cref@result] \csname the\pfcasecounter@pmg \endcsname}}{}%
        }
        \begin{pfcasesnumbered}}{\end{pfcasesnumbered}}
    }


  \ltx@ifpackageloaded{cleveref}{
    \crefformat{pfcasesnonumi}{#2case~#1#3}
    \Crefformat{pfcasesnonumi}{#2Case~#1#3}
    \crefformat{pfcasesnonumii}{#2subcase~#1#3}
    \Crefformat{pfcasesnonumii}{#2Subcase~#1#3}
    \crefformat{pfcasesnonumiii}{#2subsubcase~#1#3}
    \Crefformat{pfcasesnonumiii}{#2Subsubcase~#1#3}
    \crefformat{pfcasesnumberedi}{#2case~#1#3}
    \Crefformat{pfcasesnumberedi}{#2Case~#1#3}
    \crefformat{pfcasesnumberedii}{#2case~#1#3}
    \Crefformat{pfcasesnumberedii}{#2Case~#1#3}
    \crefformat{pfcasesnumberediii}{#2case~#1#3}
    \Crefformat{pfcasesnumberediii}{#2Case~#1#3}
    }{}
    \newlist{pfcasesnonum}{enumerate}{3}
    \setlist[pfcasesnonum]{
      label={\pmg@casefont{Case}},
      align=left,
      labelsep=0pt,
      itemindent=!,
      leftmargin=2em,
      labelindent=0pt,
      labelwidth=0pt
    }
    \setlist[pfcasesnonum,1]{
        before=\def\pfcasecounter@pmg{pfcasesnonumi},
    }
    \setlist[pfcasesnonum,2]{
        before=\def\pfcasecounter@pmg{pfcasesnonumii},
    }
    \setlist[pfcasesnonum,3]{
        before=\def\pfcasecounter@pmg{pfcasesnonumiii},
    }
    \newlist{pfcasesnumbered}{enumerate}{3}
    \setlist[pfcasesnumbered]{
      align=left,
      labelsep=0pt,
      itemindent=!,
      leftmargin=2em,
      labelindent=0pt,
      labelwidth=0pt
    }
    \setlist[pfcasesnumbered,1]{
        before=\def\pfcasecounter@pmg{pfcasesnumberedi},
      label={{\pmg@casefont{Case}}~\arabic*},
      ref={\arabic*},
    }
    \setlist[pfcasesnumbered,2]{
        before=\def\pfcasecounter@pmg{pfcasesnumberedii},
      label={{\pmg@casefont{Case}}~\arabic{pfcasesnumberedi}\alph*},
      ref={\arabic{pfcasesnumberedi}\alph*},
      labelindent=-1em,
    }
    \setlist[pfcasesnumbered,3]{
        before=\def\pfcasecounter@pmg{pfcasesnumberediii},
      label={{\pmg@casefont{Case}}~\arabic{pfcasesnumberedi}\alph{pfcasesnumberedii}.\roman*},
      ref={\arabic{pfcasesnumberedi}\alph{pfcasesnumberedii}.\roman*},
      labelindent=-1em,
    }
    \def\@recthy@cases@labelindent{0pt}
    \newenvironment{pfcases*}{
        \def\@recthy@cases@labelindent{-1em}
        \ProvideDocumentCommand{\case}{r[]}{
                \def\thiscase{~##1}%
            \item~##1\textbf{:} %
            \ltx@ifpackageloaded{cleveref}{%
            \cref@constructprefix{pfcases}{\cref@result}%
            \protected@xdef\cref@currentlabel{[\pfcasecounter@pmg][][\cref@result]##1}}{}  \protected@edef\@currentlabel{##1}\ignorespaces}
    \begin{pfcasesnonum}[labelindent=\@recthy@cases@labelindent]\def\@recthy@cases@labelindent{-1em}\ignorespaces}{\end{pfcasesnonum}\ignorespacesafterend}

    \newenvironment{pfcases}{
        \ProvideDocumentCommand{\case}{o}{
            \IfNoValueTF{##1}{
                \def\thiscase{}%
            }{
                \def\thiscase{~##1}%
            }
            \item~##1\textbf{:} \linebreak[1]%
            \ltx@ifpackageloaded{cleveref}{%
            \cref@constructprefix{\pfcasecounter@pmg}{\cref@result}%
            \protected@xdef\cref@currentlabel{[\pfcasecounter@pmg][][\cref@result]\csname the\pfcasecounter@pmg \endcsname}}{}\ignorespaces}
    
    \begin{pfcasesnumbered}[labelindent=\@recthy@cases@labelindent]\def\@recthy@cases@labelindent{-1em}\ignorespaces}{\end{pfcasesnumbered}\ignorespacesafterend}
\fi


\ifrecthy@nosubfun
    \@packinfo{Not loading subfun defs.}
\else
    \@packinfo{Loading subfun defs.}
    %DOCUMENT ME
  \newcommand{\@pnrelbar}{%
    \linethickness{\dimen2}%
    \sbox\z@{$\m@th\prec$}%
    \dimen@=1.1\ht\z@
    \begin{picture}(\dimen@,.4ex)
    \roundcap
    \put(0,.2ex){\line(1,0){\dimen@}}
    \put(\dimexpr 0.5\dimen@-.2ex\relax,0){\line(1,1){.4ex}}
    \end{picture}%
  }


  \newcommand{\@snrelbar}{%
    \linethickness{\dimen2}%
    \sbox\z@{$\m@th\succ$}%
    \dimen@=1.1\ht\z@
    \begin{picture}(\dimen@,.4ex)
    \roundcap
    \put(0,.2ex){\line(1,0){\dimen@}}
    \put(\dimexpr 0.5\dimen@-.2ex\relax,0){\line(1,1){.4ex}}
    \end{picture}%
  }


  \newcommand{\@precneq}{\mathrel{\vcenter{\hbox{\text{\prec@neq}}}}}
  \newcommand{\prec@neq}{%
    \dimen2=\f@size\dimexpr.04pt\relax
    \oalign{%
      \noalign{\kern\dimexpr.2ex-.5\dimen2\relax}
      $\m@th\prec$\cr
      \noalign{\kern-.5\dimen2}
      \hidewidth\@pnrelbar\hidewidth\cr
    }%
  }


  \newcommand{\@succneq}{\mathrel{\vcenter{\hbox{\text{\succ@neq}}}}}
  \newcommand{\succ@neq}{%
    \dimen2=\f@size\dimexpr.04pt\relax
    \oalign{%
      \noalign{\kern\dimexpr.2ex-.5\dimen2\relax}
      $\m@th\prec$\cr
      \noalign{\kern-.5\dimen2}
      \hidewidth\@snrelbar\hidewidth\cr
    }%
  }


  \NewDocumentCommand{\subfun}{o}{\prec\IfNoValueTF{#1}{}{_{#1}}}
  \NewDocumentCommand{\supfun}{o}{\succ\IfNoValueTF{#1}{}{_{#1}}}
  \NewDocumentCommand{\nsubfun}{o}{\nprec\IfNoValueTF{#1}{}{_{#1}}}
  \NewDocumentCommand{\nsupfun}{o}{\nsucc\IfNoValueTF{#1}{}{_{#1}}}
  \NewDocumentCommand{\subfuneq}{o}{\preceq\IfNoValueTF{#1}{}{_{#1}}}
  \NewDocumentCommand{\subfunneq}{o}{\@precneq\IfNoValueTF{#1}{}{_{#1}}}
  \NewDocumentCommand{\supfuneq}{o}{\succeq\IfNoValueTF{#1}{}{_{#1}}}
  \NewDocumentCommand{\supfunneq}{o}{\@succneq\IfNoValueTF{#1}{}{_{#1}}}
  \NewDocumentCommand{\nsubfuneq}{o}{\npreceq\IfNoValueTF{#1}{}{_{#1}}}
  \NewDocumentCommand{\nsupfuneq}{o}{\nsucceq\IfNoValueTF{#1}{}{_{#1}}}
\fi




\newcommand*{\@pmg@saveeqstate}{\let\@origif@fleqn@pmg=\if@fleqn \let\@origiftagsleft@pmg=\iftagsleft@}
\newcommand*{\@pmg@restoreeqstate}{ \let\if@fleqn=\@origif@fleqn@pmg\let\iftagsleft@=\@origiftagsleft@pmg}

\ifrecthy@noreqhelper
    \@packinfo{Not loading requirement assist.}
\else
    \@packinfo{Loading requirement assist.}
    % \newcommand*{\req@beamer}[3][]{\ensuremath{\@recthy@reqscr{#2}^{#1}_{#3}}}
    \newcommand*{\req@orig}[3]{\ensuremath{\@ifmtarg{#1}{\@recthy@reqscr{#2}_{#3}}{\@recthy@reqscr{#2}_{#3}^{#1}}}} 
    \NewDocumentCommand{\req@nohref}{sO{}mO{}m}{\req@orig{#2#4}{#3}{#5}}
    \NewDocumentCommand{\req@href}{O{}mO{}m}{\hyperref[\detokenize{req:#2@#1#3}]{\req@orig{#1#3}{#2}{#4}}}

   \NewDocumentCommand\req@href@star{s}{\IfBooleanTF#1{\req@nohref}{\req@href}}
   \newcommand{\req}{\req@nohref}
    \ifrecthy@hyperreqs
      \AtBeginDocument{\@ifundefined{hyperref}{}{\let\req=\req@href@star}}
    \fi
    
    \newtagform{colon}{}{:}
    % \@ifclassloaded{beamer}{\def\setmmonefivecm{\@mathmargin{1.5cm}}}{\def\setmmonefivecm{\setlength\@mathmargin{1.5cm}}}%  
    \newenvironment{requirement}[1]{
        \@pmg@saveeqstate
        \@fleqntrue
        \setlength\@mathmargin{1.5cm}
        \tagsleft@true
        \usetagform{colon}
        \begin{equation}\tag{#1}
    }{
        \end{equation}
        \@pmg@restoreeqstate
        \usetagform{default}
        % \let\maketag@@@=\maketag@@@orig@pmg  
    }

    \NewDocumentEnvironment{require}{O{}mO{}m}{
        \begin{requirement}{\req@nohref[#1#3]{#2}{#4}} \label{\detokenize{req:#2@#1#3}}%
    }{
        \end{requirement}%
    }
    \NewDocumentEnvironment{require*}{O{}mO{}m}{
        \begin{requirement}{\req@nohref[#1#3]{#2}{#4}}
    }{
        \end{requirement}%
    }



    \NewDocumentCommand{\refreq}{O{}mO{}}{\ref{\detokenize{req:#2@#1#3}}}

    \NewDocumentCommand{\require@nested}{sO{}mO{}m}{%
        \tag{\req@nohref[#2#4]{#3}{#5}}\IfBooleanTF{#1}{}{\label{\detokenize{req:#3@#2#4}}}%
    }



\ifrecthy@beamersupp
        \newtheorem{reqblock}{Requirements}
    \let\OLDreqblock=\reqblock
    \def\reqblock{%
      \setbeamercolor{block title}{fg=white,bg=Maroon}%
      \setbeamercolor{block body}{fg=black,bg=Maroon!10}\OLDreqblock
    }

      \NewDocumentCommand{\require@beamer}{sO{}mO{}m}{\tag{$\@ifmtarg{#2#4}{\mathscr{#3}_{#5}}{\mathscr{#3}_{#5}^{#2#4}}$} \IfBooleanTF{#1}{}{\label{\detokenize{req:#3@#2#4}}} & }

    \NewDocumentEnvironment{BeamerRequirements}{d<>b}{\IfValueTF{#1}{\begin{reqblock}<#1>}{\begin{reqblock}}%
    \vspace*{-.5cm}
    \usetagform{colon} %
          \let\myhalign\halign%
            \def\halign{\let\mylb\\ \renewcommand{\\}{& \mylb}\myhalign}%
          \let\require=\require@beamer %
          \begin{flalign*}%
          #2 &
          \end{flalign*}%
          \end{reqblock}}{}

    \NewDocumentEnvironment{requirements}{b}{\usetagform{colon} %
          \let\myhalign\halign%
            \def\halign{\let\mylb\\ \renewcommand{\\}{& \mylb}\myhalign}%
          \let\require=\require@beamer %
          \begin{flalign*}%
          #1 &
          \end{flalign*}}{\ignorespacesafterend}%\@recthy@ignorespacesandallparsafter}
    \else

        \NewDocumentEnvironment{requirements}{}{%
        \@pmg@saveeqstate%
        \@fleqntrue%
        \setlength\@mathmargin{1.5cm}%
        \tagsleft@true%
        \usetagform{colon}%
        \let\require=\require@nested%
        \gather
        }{\endgather%
        \@pmg@restoreeqstate%
        \usetagform{default}%
        \ignorespacesafterend}

 \fi


\fi


\ifrecthy@nostructresets
    \@packinfo{Not loading r.e. set structs w/ containment.}
\else
  \@packinfo{Loading r.e. set structs w/ containment.}
  \let\@orig@setminus=\setminus
  \newcommand*{\promptdif}{\@orig@setminus_{\downarrow}}
  \let\setbefore=\promptdif
  \let\promptminus=\promptdif
  \let\promptsetminus=\promptdif

  \newcommand*{\StrcE}{\ensuremath{\mathscr{E}}}
  \WithSuffix\def\StrcE*{\ensuremath{\mathscr{E}^{*}}}
  \let\strucE\StrcE
  \newcommand*{\StrcL}[1]{\mathscr{L}(#1)}
  \newcommand*{\StrcStarL}[1]{\mathscr{L}^{*}(#1)}
  \WithSuffix\def\StrcL*{\StrcStarL}
  \newcommand*{\splitof}{\sqsubset}
  \newcommand*{\splitby}{\sqsupset}
  \newcommand*{\StrcR}{\ensuremath{\mathcal{R}}}
  \newcommand*{\StrcD}{\ensuremath{\mathcal{D}}}


\fi


\ifrecthy@nohyphenation
    \@packinfo{No hyphenation loading.}
\else
\hyphenation{ra-tion-als ra-tion-al Harrington Slaman Cholak Knight Gerdes Lange}
\fi




\ifrecthy@mrref
    \@packinfo{MRref helper loading}
    \AtBeginDocument{
      \@ifpackageloaded{hyperref}{
        \def\recthy@checkMR MR#1#2#3 #4\relax%
          {\ifx#1M%
             \ifx#2R\recthy@MR{#3}\else\recthy@MR{#1#2#3}\fi
           \else
             \recthy@MR{#1#2#3}%
           \fi}
        \def\MR#1{\recthy@checkMR MR#1 \relax}
        \newcommand{\recthy@MR}[1]{\recthy@MRhref{#1}{MR #1}}
        % % \MRhref is called by the amsart/book/proc definition of \MR.
        \newcommand{\recthy@MRhref}[2]{\href{http://www.ams.org/mathscinet-getitem?mr=#1}{#2}}
      }
    }
\else
    \@packinfo{MRref helper not loading}
\fi


% \newcommand*{\@recthy@abs}[1]{\lvert#1\rvert}
\DeclarePairedDelimiter\@recthy@abs{\lvert}{\rvert}
\newcommand*{\@recthy@ensuretext}[1]{\ensuremath{\text{#1}}}


\ifrecthy@nomath
    \@packinfo{No math loading.}
\else
    \@packinfo{math loading.}
    \providecommand*{\eqdef}{\overset{\text{\tiny def}}{=}} %\newcommand*{\eqdef}{\ensuremath{=\limits_{\text{\tiny def}}}}
    \providecommand*{\iffdef}{\stackrel{\text{\tiny def}}{\iff}}
    \ifdefined\aut
    \else
        \DeclareMathOperator{\aut}{Aut}
    \fi
    \ifdefined\Ord
    \else
        \DeclareMathOperator{\Ord}{Ord}
    \fi
    \DeclarePairedDelimiter\ceil{\lceil}{\rceil}
    \DeclarePairedDelimiter\floor{\lfloor}{\rfloor}

    \recthy@strip{abs}
    \ifdefined\dom
    \else
        \DeclareMathOperator{\dom}{dom}
    \fi
    \ifdefined\rng
    \else
        \DeclareMathOperator{\rng}{rng}
    \fi
    \providecommand*{\restr}[1]{\mathpunct{\restriction_{#1}}}
    \providecommand*{\ordpair}[2]{\left( #1, #2 \right) }
    %%%%%%%%Function commands

    \appto{\recthy@rollback@threeTwo}{
        \providecommand*{\map}[2]{:##1\mapsto ##2}
        \providecommand*{\functo}[3]{##1\map{##2}{##3}}
        \ifdefined\funcomp
        \else
            \let\funcomp=\compfunc
        \fi
    }

    \providecommand*{\compfunc}{\mathbin{\circ}}
    \ifdefined\compose
    \else
        \let\compose=\compfunc
    \fi
    \ltx@ifpackageloaded{unicode-math}{
            \newcommand{\pmapsto}{\mathrel{\nvrightarrow}}
            \newcommand{\pmapsfrom}{\mathrel{\nvleftarrow}}
    }{
        % \newcommand{\pmapsto}{}% just for safety
        % \newcommand{\pmapsfrom}{}% just for safety

        \DeclareRobustCommand{\pmapsto}{\mathrel{\mathpalette\p@to@gets\to}}
        \DeclareRobustCommand{\pmapsfrom}{\mathrel{\mathpalette\p@to@gets\gets}}

        \newcommand{\p@to@gets}[2]{%
          \ooalign{\hidewidth$\m@th#1\mapstochar\mkern5mu$\hidewidth\cr$\m@th#1\to$\cr}%
        }
    }
    \newcommand{\fpmapsto}{\mathrel{\pmapsto_{< \infty}}}
    \newcommand{\fpmapsfrom}{\mathrel{\pmapsfrom_{< \infty}}}
    \newcommand{\ParFuncs}[2]{{#2}^{\subset #1}}
    \newcommand{\FinParFuncs}[2]{{#2}^{\subset_{< \infty} #1}}
    \recthy@strip{ensuretext}
\fi




\ifrecthy@nooperators
    \@packinfo{No operators loading.}
\else
    \@packinfo{Operators loading.}
        \providecommand*{\meet}{\mathbin{\wedge}}
    \providecommand*{\join}{\mathbin{\vee}}
    \providecommand*{\Meet}{\mathop{\bigwedge}}
    \let\MnJoin=\Join
    \renewcommand*{\Join}{\mathop{\bigvee}}
    \ifdefined\xor
    \else
        \DeclareMathOperator{\xor}{xor}
    \fi
\fi


\newcommand*{\@recthy@llangle}{\langle\!\langle}
\newcommand*{\@recthy@rrangle}{\rangle\!\rangle}



\ifrecthy@nodelim
    \@packinfo{No delimiters loading.}
\else
    \ltx@ifpackageloaded{MnSymbol}{}{ 
    \DeclareFontFamily{OMX}{MnSymbolE}{}
    \DeclareSymbolFont{MnLargeSymbols}{OMX}{MnSymbolE}{m}{n}
    \SetSymbolFont{MnLargeSymbols}{bold}{OMX}{MnSymbolE}{b}{n}
    \DeclareFontShape{OMX}{MnSymbolE}{m}{n}{
        <-6>  MnSymbolE5
       <6-7>  MnSymbolE6
       <7-8>  MnSymbolE7
       <8-9>  MnSymbolE8
       <9-10> MnSymbolE9
      <10-12> MnSymbolE10
      <12->   MnSymbolE12
    }{}
    \DeclareFontShape{OMX}{MnSymbolE}{b}{n}{
        <-6>  MnSymbolE-Bold5
       <6-7>  MnSymbolE-Bold6
       <7-8>  MnSymbolE-Bold7
       <8-9>  MnSymbolE-Bold8
       <9-10> MnSymbolE-Bold9
      <10-12> MnSymbolE-Bold10
      <12->   MnSymbolE-Bold12
    }{}
    \DeclareMathDelimiter{\@recthy@ulcorner}
        {\mathopen}{MnLargeSymbols}{'036}{MnLargeSymbols}{'036}
    \DeclareMathDelimiter{\@recthy@urcorner}
        {\mathclose}{MnLargeSymbols}{'043}{MnLargeSymbols}{'043}
        \let\ulcorner=\@recthy@ulcorner
        \let\urcorner=\@recthy@urcorner
}
    \newcommand*{\godelnum}[1]{\left\ulcorner #1 \right\urcorner}
    \let\cornerquote=\godelnum
    \let\gcode=\godelnum
    \let\@recthy@llangle@orig=\llangle
    \let\@recthy@rrangle@orig=\rrangle
    \ifrecthy@nodoubleangles
    \else
      \let\llangle=\@recthy@llangle
      \let\rrangle=\@recthy@rrangle
    \fi
\fi


\newcommand*{\@recthy@re}{\@recthy@ensuretext{r.e.\ }}
\newcommand*{\@recthy@ce}{\@recthy@ensuretext{c.e.\ }}
\newcommand*{\@recthy@RE}{\@recthy@ensuretext{R.E.\ }}
\newcommand*{\@recthy@CE}{\@recthy@ensuretext{C.E.\ }}
\newcommand*{\@recthy@Re}{\@recthy@ensuretext{R.e.\ }}
\newcommand*{\@recthy@Ce}{\@recthy@ensuretext{C.e.\ }}
\newcommand*{\@recthy@recursive}{\@recthy@ensuretext{recursive}}
\newcommand*{\@recthy@computable}{\@recthy@ensuretext{computable}}
\newcommand*{\@recthy@recursivelyEnumerable}{\@recthy@ensuretext{recursively enumerable}}
\newcommand*{\@recthy@computablyEnumerable}{\@recthy@ensuretext{computably enumerable}}
\newcommand*{\@recthy@Recursive}{\@recthy@ensuretext{Recursive}}
\newcommand*{\@recthy@Computable}{\@recthy@ensuretext{Computable}}
\newcommand*{\@recthy@RecursivelyEnumerable}{\@recthy@ensuretext{Recursively enumerable}}
\newcommand*{\@recthy@ComputablyEnumerable}{\@recthy@ensuretext{Computably enumerable}}
\def\@recthy@REA{REA}
\def\@recthy@CEA{CEA}
% \mathchardef\@recthy@mhyphen="2D % Define a "math hyphen"
\def\@recthy@mhyphen{\mathhyphen}
\NewDocumentCommand{\REA}{d()o}{\ensuremath{\IfValueTF{#2}{#2 \@recthy@mhyphen}{}\text{\@recthy@REA}\IfValueTF{#1}{(#1)}{}}}
\NewDocumentCommand{\CEA}{d()o}{\ensuremath{\IfValueTF{#2}{#2 \@recthy@mhyphen}{}\text{\@recthy@CEA}\IfValueTF{#1}{(#1)}{}}}


\ifrecthy@nonames
    \@packinfo{No names loading.}
\else
    \@packinfo{Names loading.}
    \recthy@strip{re}
    \recthy@strip{ce}
    \recthy@strip{RE}
    \recthy@strip{CE}
    \recthy@strip{Re}
    \recthy@strip{Ce}
    \recthy@strip{recursive}
    \recthy@strip{computable}
    \recthy@strip{recursivelyEnumerable}
    \recthy@strip{computablyEnumerable}
    \recthy@strip{Recursive}
    \recthy@strip{Computable}
    \recthy@strip{RecursivelyEnumerable}
    \recthy@strip{ComputablyEnumerable}
\fi



% Set stuff



\NewDocumentCommand{\@recthy@set}{m!G{}}{{\let\st=\mid\left\{#1 \ifthenelse{\isempty{#2}}{}{\mid #2} \right\}}}

% \def\@recthy@set#1{\def\@set@argi@pmg{#1}\@ifnextchar\bgroup\@setst@pmg\@setplain@pmg}
% \newcommand*{\@setst@pmg}[1]{\ifthenelse{\isempty{#1}}{\@setplain@pmg}{\left \{ \@set@argi@pmg \middle | #1\right \}}}
% \newcommand*{\@setplain@pmg}{\left \{ \@set@argi@pmg \right \}}
% \newcommand*{\@recthy@set}[2]{\ifthenelse{\isempty{#2}}{\left \{  #1 \right \}}{\left \{ #1 \middle | #2\right \}} }
% \DeclareOptionX{setminussym}[@recthy@defaultsetminusSYM]{\edef\@recthy@setminusSYM{\expandafter\noexpand\csname #1 \endcsname}}


\newcommand*{\@recthy@card}[1]{\lvert#1\rvert}
\newcommand*{\@recthy@union}{\mathbin{\cup}}
\newcommand*{\@recthy@Union}{\mathop{\bigcup}}
\newcommand*{\@recthy@isect}{\mathbin{\cap}}
\newcommand*{\@recthy@Isect}{\mathop{\bigcap}}
\newcommand*{\@recthy@nin}{\notin}
\NewDocumentCommand{\@recthy@powset}{O{}m}{\mathcal{P}_{#1}\left(#2\right)}
\NewDocumentCommand{\@recthy@finSsets}{r[]E{^}{{< \omega}}}{\left[#1\right]^{#2}}
\newcommand*{\@recthy@eset}{\emptyset}
\newcommand*{\@recthy@setovercmp}[1]{\overline{#1}}
\newcommand*{\@recthy@setsimcmp}[1]{\backsim #1}
\let\@recthy@setcmp=\@recthy@setovercmp
\newcommand*{\@recthy@setdiff}{\mathbin{\@recthy@setdiffSYM}}







% \WithSuffix\def\@recthy@setcmp[#1]#2{#1\@recthy@setminus#2}
\ifrecthy@nosets
    \@packinfo{No set cmds loading.}
\else
    \@packinfo{Set cmds loading.}
    \recthy@strip{set}
    \recthy@strip{card}
    \recthy@strip{union}
    \recthy@strip{Union}
    \recthy@strip{isect}
    \recthy@strip{Isect}
    \recthy@strip{powset}
    \recthy@strip{finSsets}
    \let\ssetsOfsize\finSsets
    \newcommand*{\finsets}{\finSsets[\omega]\relax}
    \recthy@strip{eset}
    \recthy@strip{nin}
    \recthy@strip{setcmp}
    % \recthy@strip{setovercmp}
    % \recthy@strip{setminuscmp}
    \recthy@strip{setdiff}
    \let\crossOrig=\cross
    \def\cross{}
    \renewcommand*{\cross}{\mathbin{\times}}
    \let\CrossOrig=\Cross
    \def\Cross{}
    \renewcommand*{\Cross}{\mathop{\Pi}}
    \providecommand*{\symdiff}{\mathbin{\Delta}}
    % \ifdefined\interior
    % \else
    %     \DeclareMathOperator{\interior}{int}
    % \fi
    % \ifdefined\closure
    % \else
        % \DeclareMathOperator{\clos}{cl}
    % \fi
\fi



\ifrecthy@noquants
    \@packinfo{No quants loading.}
\else
    \@packinfo{Quants loading.}
    \AtBeginDocument{
        \let\exists@orig@recthy=\exists
        \let\forall@orig@recthy=\forall
        \newcommand*{\exists@recthy}{\exists@orig@recthy}
        \newcommand*{\forall@recthy}{\forall@orig@recthy}
        % \NewDocumentCommand{\exists@paren@recthy}{O{}m}{\left(\exists@orig@recthy^{#1} #2 \right)\!}
        % \NewDocumentCommand{\exists@brak@recthy}{O{}m}{\left[\exists@orig@recthy^{#1} #2 \right]\!}
        \let\nexists@orig@recthy=\nexists
        \newcommand*{\nexists@recthy}{\nexists@orig@recthy}
        \let\nforall@orig@recthy=\nforall
        \newcommand*{\nforall@recthy}{\nforall@orig@recthy}
        \newcommand*{\existsuniq}{\exists@orig@recthy !}
        \newcommand*{\nexistsuniq}{\nexists@orig@recthy !}

        \WithSuffix\def\existsuniq(#1){\left(\existsuniq #1 \right)\!}
        \WithSuffix\def\nexistsuniq(#1){\left(\nexistsuniq #1 \right)\!}
        \WithSuffix\def\existsuniq[#1]{\left[\existsuniq #1 \right]\!}
        \WithSuffix\def\nexistsuniq[#1]{\left[\nexistsuniq #1 \right]\!}

        \newcommand*{\forallae}{\forall@orig@recthy^{*}}
        \WithSuffix\def\forallae(#1){\left(\forallae #1\right)\!}
        \WithSuffix\def\forallae[#1]{\left[\forallae #1\right]\!}

        \newcommand*{\nforallae}{\nforall@orig@recthy^{*}}
        \WithSuffix\def\nforallae(#1){\left(\nforallae #1\right)\!}
        \WithSuffix\def\nforallae[#1]{\left[\nforallae #1\right]\!}

        \providecommand*{\existsinf}{\exists@orig@recthy^{\infty}}
        \WithSuffix\def\existsinf(#1){\left(\existsinf #1\right)\!}
        \WithSuffix\def\existsinf[#1]{\left[\existsinf #1\right]\!}

        \providecommand*{\nexistsinf}{\nexists^{\infty}}
        \WithSuffix\def\nexistsinf(#1){\left(\nexistsinf #1\right)\!}
        \WithSuffix\def\nexistsinf[#1]{\left[\nexistsinf #1\right]\!}


        \WithSuffix\def\exists@recthy(#1){\left(\exists #1 \right)\!}
        % \NewDocumentCommand{\exists@recthy}{d()od()o}{\IfValueTF{#1}{\exists@paren@recthy{#1}}{\IfValueTF{#1}{\exists@brak@recthy{#2}} } }
        \WithSuffix\def\forall@recthy(#1){\left(\forall #1 \right)\!}
        \WithSuffix\def\nexists@recthy(#1){\left(\nexists #1 \right)\!}
        \WithSuffix\def\exists@recthy[#1]{\left[\exists #1 \right]\!}
        \WithSuffix\def\forall@recthy[#1]{\left[\forall #1 \right]\!}
        \WithSuffix\def\nexists@recthy[#1]{\left[\nexists #1 \right]\!}
        \WithSuffix\def\forall@recthy*{\forallae}
        \WithSuffix\def\nforall@recthy*{\nforallae}
        \WithSuffix\def\exists@recthy*{\existsinf}
        \WithSuffix\def\nexists@recthy*{\nexistsinf}
        \WithSuffix\def\exists@recthy!{\existsuniq}
        \let\exists=\exists@recthy
        \let\forall=\forall@recthy
        \let\nexists=\nexists@recthy
        \let\nforall=\nforall@recthy
    }



    \providecommand*{\True}{\top}
    \providecommand*{\False}{\bot}

    \providecommand*{\Land}{\mathop{\bigwedge}}
    \providecommand*{\Lor}{\mathop{\bigvee}}
    \providecommand*{\LLand}{\mathop{\bigwedge\mkern-15mu\bigwedge}}
    \providecommand*{\LLor}{\mathop{\bigvee\mkern-15mu\bigvee}}

    \providecommand*{\liff}{\ensuremath{\leftrightarrow}}
    \providecommand*{\limplies}{\ensuremath{\rightarrow}}

\fi

%%%%%%%%% Formula Classes




\ifrecthy@nosyntax
    \@packinfo{No syntax loading.}
\else
    \@packinfo{syntax loading.}
    \recthy@strip{CDelta}
    \recthy@strip{CSigma}
    \recthy@strip{CPi}
    \newcommand*{\deltan}[1]{\Delta_{#1}}
    \WithSuffix\def\deltan[#1]#2{\Delta^{#1}_{#2}}
    % \WithSuffix\def\deltan(#1)#2{\deltan{#2}\!\left(#1\right)}
    \newcommand*{\deltaZeroN}[1]{\Delta^{0}_{#1}}
    \WithSuffix\def\deltaZeroN[#1]#2{\Delta^{0,#1}_{#2}}
    % \WithSuffix\def\deltaZeroN(#1)#2{\deltaZeroN{#2}\!\left(#1\right)}
    \newcommand*{\deltaZeroTwo}{\deltaZeroN{2}}
    \WithSuffix\def\deltaZeroTwo[#1]{\deltaZeroN[#1]{2}}
    % \WithSuffix\def\deltaZeroTwo(#1){\deltaZeroN[#1]{2}}
    \newcommand*{\deltaZeroThree}{\deltaZeroN{3}}
    \WithSuffix\def\deltaZeroThree[#1]{\deltaZeroN[#1]{3}}
    % \WithSuffix\def\deltaZeroThree(#1){\deltaZeroN[#1]{3}}
    \newcommand*{\deltaZeroOne}{\deltaZeroN{1}}
    \WithSuffix\def\deltaZeroOne[#1]{\deltaZeroN[#1]{1}}
    % \WithSuffix\def\deltaZeroOne(#1){\deltaZeroN[#1]{1}}
    \newcommand*{\deltaZeroZero}{\deltaZeroN{0}}
    \WithSuffix\def\deltaZeroZero[#1]{\deltaZeroN[#1]{0}}
    % \WithSuffix\def\deltaZeroZero(#1){\deltaZeroN[#1]{0}}


    \newcommand*{\deltaOneN}[1]{\deltan{#1}^1}
    \WithSuffix\def\deltaOneN[#1]#2{\Delta^{1,#1}_{#2}}
    % \WithSuffix\def\deltaOneN(#1)#2{\deltaOneN{#2}\!\left(#1\right)}
    \newcommand*{\deltaOneOne}{\deltaOneN{1}}
    \WithSuffix\def\deltaOneOne[#1]{\deltaOneN[#1]{1}}
    % \WithSuffix\def\deltaOneOne(#1){\deltaOneN[#1]{1}}
    \newcommand*{\deltaOneTwo}{\deltaOneN{2}}
    \WithSuffix\def\deltaOneTwo[#1]{\deltaOneN[#1]{2}}
    % \WithSuffix\def\deltaOneTwo(#1){\deltaOneN[#1]{2}}
    \newcommand*{\deltaOneThree}{\deltaOneN{3}}
    \WithSuffix\def\deltaOneThree[#1]{\deltaOneN[#1]{3}}
    % \WithSuffix\def\deltaOneThree(#1){\deltaOneN[#1]{3}}



    \newcommand*{\sigman}[1]{\Sigma_{#1}}
    \WithSuffix\def\sigman[#1]#2{\Sigma^{#1}_{#2}}
    % \WithSuffix\def\sigman(#1)#2{\sigman{#2}\!\left(#1\right)}
    \newcommand*{\sigmaZeroN}[1]{\Sigma^{0}_{#1}}
    \WithSuffix\def\sigmaZeroN[#1]#2{\Sigma^{0,#1}_{#2}}
    % \WithSuffix\def\sigmaZeroN(#1)#2{\sigmaZeroN{#2}\!\left(#1\right)}
    \newcommand*{\sigmaZeroOne}{\sigmaZeroN{1}}
    \WithSuffix\def\sigmaZeroOne[#1]{\sigmaZeroN[#1]{1}}
    % \WithSuffix\def\sigmaZeroOne(#1){\sigmaZeroN[#1]{1}}
    \newcommand*{\sigmaZeroTwo}{\sigmaZeroN{2}}
    \WithSuffix\def\sigmaZeroTwo[#1]{\sigmaZeroN[#1]{2}}
    % \WithSuffix\def\sigmaZeroTwo(#1){\sigmaZeroN[#1]{2}}
    \newcommand*{\sigmaZeroThree}{\sigmaZeroN{3}}
    \WithSuffix\def\sigmaZeroThree[#1]{\sigmaZeroN[#1]{3}}
    % \WithSuffix\def\sigmaZeroThree(#1){\sigmaZeroN[#1]{3}}



    \newcommand*{\sigmaOneN}[1]{\Sigma^1_{#1}}
    \WithSuffix\def\sigmaOneN[#1]#2{\Sigma^{1,#1}_{#2}}
    % \WithSuffix\def\sigmaOneN(#1)#2{\sigmaOneN{#2}\!\left(#1\right)}
    \newcommand*{\sigmaOneOne}{\sigmaOneN{1}}
    \WithSuffix\def\sigmaOneOne[#1]{\sigmaOneN[#1]{1}}
    % \WithSuffix\def\sigmaOneOne(#1){\sigmaOneN[#1]{1}}
    \newcommand*{\sigmaOneTwo}{\sigmaOneN{2}}
    \WithSuffix\def\sigmaOneTwo[#1]{\sigmaOneN[#1]{2}}
    % \WithSuffix\def\sigmaOneTwo(#1){\sigmaOneN[#1]{2}}
    \newcommand*{\sigmaOneThree}{\sigmaOneN{3}}
    \WithSuffix\def\sigmaOneThree[#1]{\sigmaOneN[#1]{3}}
    % \WithSuffix\def\sigmaOneThree(#1){\sigmaOneN[#1]{3}}



    \newcommand*{\pin}[1]{\Pi_{#1}}
    \WithSuffix\def\pin[#1]#2{\Pi^{#1}_{#2}}
    % \WithSuffix\def\pin(#1)#2{\pin{#2}\!\left(#1\right)}
    \newcommand*{\piZeroN}[1]{\Pi^0_{#1}}
    \WithSuffix\def\piZeroN[#1]#2{\Pi^{0,#1}_{#2}}
    % \WithSuffix\def\piZeroN(#1)#2{\piZeroN{#2}\left(#1\right)}
    \newcommand*{\piZeroOne}{\piZeroN{1}}
    \WithSuffix\def\piZeroOne[#1]{\piZeroN[#1]{1}}
    % \WithSuffix\def\piZeroOne(#1){\piZeroN[#1]{1}}
    \newcommand*{\piZeroTwo}{\piZeroN{2}}
    \WithSuffix\def\piZeroTwo[#1]{\piZeroN[#1]{2}}
    % \WithSuffix\def\piZeroTwo(#1){\piZeroN[#1]{2}}
    \newcommand*{\piZeroThree}{\piZeroN{3}}
    \WithSuffix\def\piZeroThree[#1]{\piZeroN[#1]{3}}
    % \WithSuffix\def\piZeroThree(#1){\piZeroN[#1]{3}}

    \newcommand*{\piOneN}[1]{\Pi^1_{#1}}
    \WithSuffix\def\piOneN[#1]#2{\Pi^{1,#1}_{#2}}
    % \WithSuffix\def\piOneN(#1)#2{\piOneN{#2}\!\left(#1\right)}
    \newcommand*{\piOneOne}{\piOneN{1}}
    \WithSuffix\def\piOneOne[#1]{\piOneN[#1]{1}}
    % \WithSuffix\def\piOneOne(#1){\piOneN[#1]{1}}
    \newcommand*{\piOneTwo}{\piOneN{2}}
    \WithSuffix\def\piOneTwo[#1]{\piOneN[#1]{2}}
    % \WithSuffix\def\piOneTwo(#1){\piOneN[#1]{2}}
    \newcommand*{\piOneThree}{\piOneN{3}}
    \WithSuffix\def\piOneThree[#1]{\piOneN[#1]{3}}
    % \WithSuffix\def\piOneThree(#1){\piOneN[#1]{3}}



    %%%ABREVIATIONS
    \newcommand*{\deltaz}{\deltaN{0}}
    \newcommand*{\deltazn}{\deltaZeroN}
    \newcommand*{\deltazii}{\deltaZeroTwo}
    \newcommand*{\deltazi}{\deltaZeroOne}
    \newcommand*{\deltazz}{\deltaZeroZero}
    \newcommand*{\deltaziii}{\deltaZeroThree}
    \newcommand*{\deltaIn}{\deltaOneN}
    \newcommand*{\deltaIi}{\deltaOneOne}
    \newcommand*{\deltaoneone}{\deltaOneOne}
    \newcommand*{\deltaIii}{\deltaOneTwo}
    \newcommand*{\deltaIiii}{\deltaOneThree}
    \newcommand*{\sigmazn}{\sigmaZeroN}
    \newcommand*{\sigmazi}{\sigmaZeroOne}
    \newcommand*{\sigmazii}{\sigmaZeroTwo}
    \newcommand*{\sigmaziii}{\sigmaZeroThree}
    \newcommand*{\sigmaIn}{\sigmaOneN}
    \newcommand*{\sigmaIi}{\sigmaOneOne}
    \newcommand*{\sigmaIii}{\sigmaOneTwo}
    \newcommand*{\sigmaIiii}{\sigmaOneThree}
    \newcommand*{\pizn}{\piZeroN}
    \newcommand*{\pizi}{\piZeroOne}
    \newcommand*{\pizii}{\piZeroTwo}
    \newcommand*{\piziii}{\piZeroThree}
    \newcommand*{\piIn}{\piOneN}
    \newcommand*{\piIi}{\piOneOne}
    \newcommand*{\piIii}{\piOneTwo}
    \newcommand*{\piIiii}{\piOneThree}

    %%%%%%%%%%%%%
    % \newcommand*{\sigmapizn}[1]{\sigmazn{#1} \union \pizn{#1}}
    % \WithSuffix\def\sigmapizn[#1]#2{\sigmazn[#1]{#2} \union \pizn[#1]{#2}}
    % \WithSuffix\def\sigmapizn(#1)#2{\sigmazn[#1]{#2} \union \pizn[#1]{#2}}
    % \let\pisigmazn=\sigmapizn
    %%%%%%%%%%%%%%

    %%%%%%%%
    % Logics
    %\newcommand*{\equiv}
    \newcommand*{\logic}[2]{\mathcal{L}_{#1,#2}}
    % \newcommand*{\logicwiw}{\logic{\omega_{1}}{\omega}}
    % \newcommand*{\Clogic}[2]{\mathcal{CL}_{#1,#2}}
    % \newcommand*{\FOlogic}{\logic{\omega}{\omega}}
    % \NewDocumentCommand{\hyplogic}{o}{\Clogic{\omega_{1}}{\omega}\ifthenelse{\isempty{#1}}{}{\left(#1\right)} }



    %May never use...specific notation for computable formulas...supplanted by \( \sigmazn{\alpha} \) notation.
    %%%%%%%%%%%





    \newcommand*{\Cdeltan}[1]{\CDelta_{#1}}
    \WithSuffix\def\Cdeltan[#1]#2{\Cdeltan{#2}^{#1}}
    \newcommand*{\Cpin}[1]{\CPi_{#1}}
    \WithSuffix\def\Cpin[#1]#2{\Cpin{#2}^{#1}}
    \newcommand*{\Csigman}[1]{\CSigma_{#1}}
    \WithSuffix\def\Csigman[#1]#2{\CSigma_{#2}^{#1}}

        \newcommand*{\CdeltaZeroN}[1]{\Cdeltan[0]{#1}}
    \newcommand*{\CdeltaZeroTwo}{\CdeltaZeroN{2}}
    \newcommand*{\CdeltaZeroThree}{\CdeltaZeroN{3}}
    \newcommand*{\CdeltaZeroOne}{\CdeltaZeroN{1}}
    \newcommand*{\CdeltaZeroZero}{\CdeltaZeroN{0}}
    \newcommand*{\CdeltaOneN}[1]{\Cdeltan[1]{#1}}
    \newcommand*{\CdeltaOneOne}{\CdeltaOneN{1}}
    \newcommand*{\CdeltaOneTwo}{\CdeltaOneN{2}}
    \newcommand*{\CdeltaOneThree}{\CdeltaOneN{3}}
        \newcommand*{\CpiZeroN}[1]{\Cpin[0]{#1}}
    \newcommand*{\CpiOneN}[1]{\Cpin[1]{#1}}
    \newcommand*{\CpiOneOne}{\CpiOneN{1}}
    \newcommand*{\CpiOneThree}{\CpiOneN{3}}
    \newcommand*{\CpiOneTwo}{\CpiOneN{2}}
    \newcommand*{\CpiZeroOne}{\CpiZeroN{1}}
    \newcommand*{\CpiZeroThree}{\CpiZeroN{3}}
    \newcommand*{\CpiZeroTwo}{\CpiZeroN{2}}
    \newcommand*{\CsigmaOneN}[1]{\Csigman[1]{#1}}
    \newcommand*{\CsigmaOneOne}{\CsigmaOneN{1}}
    \newcommand*{\CsigmaOneThree}{\CsigmaOneN{3}}
    \newcommand*{\CsigmaOneTwo}{\CsigmaOneN{2}}
    \newcommand*{\CsigmaZeroN}[1]{\Csigman[0]{#1}}
    \newcommand*{\CsigmaZeroOne}{\CsigmaZeroN{1}}
    \newcommand*{\CsigmaZeroThree}{\CsigmaZeroN{3}}
    \newcommand*{\CsigmaZeroTwo}{\CsigmaZeroN{2}}


    \WithSuffix\def\CsigmaOneN[#1]#2{\Csigma[1,#1]{#2}}
    \WithSuffix\def\CsigmaOneOne[#1]{\CsigmaOneN[#1]{1}}
    \WithSuffix\def\CsigmaOneThree[#1]{\CsigmaOneN[#1]{3}}
    \WithSuffix\def\CsigmaOneTwo[#1]{\CsigmaOneN[#1]{2}}
    \WithSuffix\def\CsigmaZeroN[#1]#2{\Csigma[0,#1]{#2}}
    \WithSuffix\def\CsigmaZeroOne[#1]{\CsigmaZeroN[#1]{1}}
    \WithSuffix\def\CsigmaZeroThree[#1]{\CsigmaZeroN[#1]{3}}
    \WithSuffix\def\CsigmaZeroTwo[#1]{\CsigmaZeroN[#1]{2}}
        \WithSuffix\def\CpiOneN[#1]#2{\Cpi[1,#1]{#2}}
    \WithSuffix\def\CpiOneOne[#1]{\CpiOneN[#1]{1}}
    \WithSuffix\def\CpiOneThree[#1]{\CpiOneN[#1]{3}}
    \WithSuffix\def\CpiOneTwo[#1]{\CpiOneN[#1]{2}}
    \WithSuffix\def\CpiZeroN[#1]#2{\Cpi[0,#1]{#2}}
    \WithSuffix\def\CpiZeroOne[#1]{\CpiZeroN[#1]{1}}
    \WithSuffix\def\CpiZeroThree[#1]{\CpiZeroN[#1]{3}}
    \WithSuffix\def\CpiZeroTwo[#1]{\CpiZeroN[#1]{2}}
        \WithSuffix\def\CdeltaOneN[#1]#2{\Cdelta[1,#1]{#2}}
    \WithSuffix\def\CdeltaOneOne[#1]{\CdeltaOneN[#1]{1}}
    \WithSuffix\def\CdeltaOneTwo[#1]{\CdeltaOneN[#1]{2}}
    \WithSuffix\def\CdeltaOneThree[#1]{\CdeltaOneN[#1]{3}}
        \WithSuffix\def\CdeltaZeroZero[#1]{\CdeltaZeroN[#1]{0}}
    \WithSuffix\def\CdeltaZeroN[#1]#2{\Cdelta[0,#1]{#2}}
        \WithSuffix\def\CdeltaZeroTwo[#1]{\CdeltaZeroN[#1]{2}}
        \WithSuffix\def\CdeltaZeroThree[#1]{\CdeltaZeroN[#1]{3}}
        \WithSuffix\def\CdeltaZeroOne[#1]{\CdeltaZeroN[#1]{1}}


    %%%ABREVIATIONS
    \newcommand*{\Cdeltaz}{\CdeltaN{0}}
    \newcommand*{\Cdeltazn}{\CdeltaZeroN}
    \newcommand*{\Cdeltazii}{\CdeltaZeroTwo}
    \newcommand*{\Cdeltazi}{\CdeltaZeroOne}
    \newcommand*{\Cdeltazz}{\CdeltaZeroZero}
    \newcommand*{\Cdeltaziii}{\CdeltaZeroThree}
    \newcommand*{\CdeltaIn}{\CdeltaOneN}
    \newcommand*{\CdeltaIi}{\CdeltaOneOne}
    \newcommand*{\Cdeltaoneone}{\CdeltaOneOne}
    \newcommand*{\CdeltaIii}{\CdeltaOneTwo}
    \newcommand*{\CdeltaIiii}{\CdeltaOneThree}
    \newcommand*{\Csigmazn}{\CsigmaZeroN}
    \newcommand*{\Csigmazi}{\CsigmaZeroOne}
    \newcommand*{\Csigmazii}{\CsigmaZeroTwo}
    \newcommand*{\Csigmaziii}{\CsigmaZeroThree}
    \newcommand*{\CsigmaIn}{\CsigmaOneN}
    \newcommand*{\CsigmaIi}{\CsigmaOneOne}
    \newcommand*{\CsigmaIii}{\CsigmaOneTwo}
    \newcommand*{\CsigmaIiii}{\CsigmaOneThree}
    \newcommand*{\Cpizn}{\CpiZeroN}
    \newcommand*{\Cpizi}{\CpiZeroOne}
    \newcommand*{\Cpizii}{\CpiZeroTwo}
    \newcommand*{\Cpiziii}{\CpiZeroThree}
    \newcommand*{\CpiIn}{\CpiOneN}
    \newcommand*{\CpiIi}{\CpiOneOne}
    \newcommand*{\CpiIii}{\CpiOneTwo}
    \newcommand*{\CpiIiii}{\CpiOneThree}

%%%BOLDFACE

    \newcommand*{\Deltan}[1]{\utilde{\symbf{\Delta}}_{#1}}
    \WithSuffix\def\Deltan[#1]#2{\utilde{\symbf{\Delta}}^{#1}_{#2}}
    % \WithSuffix\def\Deltan(#1)#2{\deltan{#2}\!\left(#1\right)}
    \newcommand*{\DeltaZeroN}[1]{\utilde{\symbf{\Delta}}^{0}_{#1}}
    \WithSuffix\def\DeltaZeroN[#1]#2{\utilde{\symbf{\Delta}}^{0,#1}_{#2}}
    % \WithSuffix\def\DeltaZeroN(#1)#2{\DeltaZeroN{#2}\!\left(#1\right)}
    \newcommand*{\DeltaZeroTwo}{\DeltaZeroN{2}}
    \WithSuffix\def\DeltaZeroTwo[#1]{\DeltaZeroN[#1]{2}}
    % \WithSuffix\def\DeltaZeroTwo(#1){\DeltaZeroN[#1]{2}}
    \newcommand*{\DeltaZeroThree}{\DeltaZeroN{3}}
    \WithSuffix\def\DeltaZeroThree[#1]{\DeltaZeroN[#1]{3}}
    % \WithSuffix\def\DeltaZeroThree(#1){\DeltaZeroN[#1]{3}}
    \newcommand*{\DeltaZeroOne}{\DeltaZeroN{1}}
    \WithSuffix\def\DeltaZeroOne[#1]{\DeltaZeroN[#1]{1}}
    % \WithSuffix\def\DeltaZeroOne(#1){\DeltaZeroN[#1]{1}}
    \newcommand*{\DeltaZeroZero}{\DeltaZeroN{0}}
    \WithSuffix\def\DeltaZeroZero[#1]{\DeltaZeroN[#1]{0}}
    % \WithSuffix\def\DeltaZeroZero(#1){\DeltaZeroN[#1]{0}}


    \newcommand*{\DeltaOneN}[1]{\Deltan{#1}^1}
    \WithSuffix\def\DeltaOneN[#1]#2{\utilde{\symbf{\Delta}}^{1,#1}_{#2}}
    % \WithSuffix\def\DeltaOneN(#1)#2{\DeltaOneN{#2}\!\left(#1\right)}
    \newcommand*{\DeltaOneOne}{\DeltaOneN{1}}
    \WithSuffix\def\DeltaOneOne[#1]{\DeltaOneN[#1]{1}}
    % \WithSuffix\def\DeltaOneOne(#1){\DeltaOneN[#1]{1}}
    \newcommand*{\DeltaOneTwo}{\DeltaOneN{2}}
    \WithSuffix\def\DeltaOneTwo[#1]{\DeltaOneN[#1]{2}}
    % \WithSuffix\def\DeltaOneTwo(#1){\DeltaOneN[#1]{2}}
    \newcommand*{\DeltaOneThree}{\DeltaOneN{3}}
    \WithSuffix\def\DeltaOneThree[#1]{\DeltaOneN[#1]{3}}
    % \WithSuffix\def\DeltaOneThree(#1){\DeltaOneN[#1]{3}}



    \newcommand*{\Sigman}[1]{\utilde{\symbf{\Sigma}}_{#1}}
    \WithSuffix\def\Sigman[#1]#2{\utilde{\symbf{\Sigma}}^{#1}_{#2}}
    % \WithSuffix\def\Sigman(#1)#2{\sigman{#2}\!\left(#1\right)}
    \newcommand*{\SigmaZeroN}[1]{\utilde{\symbf{\Sigma}}^{0}_{#1}}
    \WithSuffix\def\SigmaZeroN[#1]#2{\utilde{\symbf{\Sigma}}^{0,#1}_{#2}}
    % \WithSuffix\def\SigmaZeroN(#1)#2{\SigmaZeroN{#2}\!\left(#1\right)}
    \newcommand*{\SigmaZeroOne}{\SigmaZeroN{1}}
    \WithSuffix\def\SigmaZeroOne[#1]{\SigmaZeroN[#1]{1}}
    % \WithSuffix\def\SigmaZeroOne(#1){\SigmaZeroN[#1]{1}}
    \newcommand*{\SigmaZeroTwo}{\SigmaZeroN{2}}
    \WithSuffix\def\SigmaZeroTwo[#1]{\SigmaZeroN[#1]{2}}
    % \WithSuffix\def\SigmaZeroTwo(#1){\SigmaZeroN[#1]{2}}
    \newcommand*{\SigmaZeroThree}{\SigmaZeroN{3}}
    \WithSuffix\def\SigmaZeroThree[#1]{\SigmaZeroN[#1]{3}}
    % \WithSuffix\def\SigmaZeroThree(#1){\SigmaZeroN[#1]{3}}



    \newcommand*{\SigmaOneN}[1]{\utilde{\symbf{\Sigma}}^1_{#1}}
    \WithSuffix\def\SigmaOneN[#1]#2{\utilde{\symbf{\Sigma}}^{1,#1}_{#2}}
    % \WithSuffix\def\SigmaOneN(#1)#2{\SigmaOneN{#2}\!\left(#1\right)}
    \newcommand*{\SigmaOneOne}{\SigmaOneN{1}}
    \WithSuffix\def\SigmaOneOne[#1]{\SigmaOneN[#1]{1}}
    % \WithSuffix\def\SigmaOneOne(#1){\SigmaOneN[#1]{1}}
    \newcommand*{\SigmaOneTwo}{\SigmaOneN{2}}
    \WithSuffix\def\SigmaOneTwo[#1]{\SigmaOneN[#1]{2}}
    % \WithSuffix\def\SigmaOneTwo(#1){\SigmaOneN[#1]{2}}
    \newcommand*{\SigmaOneThree}{\SigmaOneN{3}}
    \WithSuffix\def\SigmaOneThree[#1]{\SigmaOneN[#1]{3}}
    % \WithSuffix\def\SigmaOneThree(#1){\SigmaOneN[#1]{3}}



    \newcommand*{\Pin}[1]{\utilde{\symbf{\Pi}}_{#1}}
    \WithSuffix\def\Pin[#1]#2{\utilde{\symbf{\Pi}}^{#1}_{#2}}
    % \WithSuffix\def\Pin(#1)#2{\pin{#2}\!\left(#1\right)}
    \newcommand*{\PiZeroN}[1]{\utilde{\symbf{\Pi}}^0_{#1}}
    \WithSuffix\def\PiZeroN[#1]#2{\utilde{\symbf{\Pi}}^{0,#1}_{#2}}
    % \WithSuffix\def\PiZeroN(#1)#2{\PiZeroN{#2}\left(#1\right)}
    \newcommand*{\PiZeroOne}{\PiZeroN{1}}
    \WithSuffix\def\PiZeroOne[#1]{\PiZeroN[#1]{1}}
    % \WithSuffix\def\PiZeroOne(#1){\PiZeroN[#1]{1}}
    \newcommand*{\PiZeroTwo}{\PiZeroN{2}}
    \WithSuffix\def\PiZeroTwo[#1]{\PiZeroN[#1]{2}}
    % \WithSuffix\def\PiZeroTwo(#1){\PiZeroN[#1]{2}}
    \newcommand*{\PiZeroThree}{\PiZeroN{3}}
    \WithSuffix\def\PiZeroThree[#1]{\PiZeroN[#1]{3}}
    % \WithSuffix\def\PiZeroThree(#1){\PiZeroN[#1]{3}}

    \newcommand*{\PiOneN}[1]{\utilde{\symbf{\Pi}}^1_{#1}}
    \WithSuffix\def\PiOneN[#1]#2{\utilde{\symbf{\Pi}}^{1,#1}_{#2}}
    % \WithSuffix\def\PiOneN(#1)#2{\PiOneN{#2}\!\left(#1\right)}
    \newcommand*{\PiOneOne}{\PiOneN{1}}
    \WithSuffix\def\PiOneOne[#1]{\PiOneN[#1]{1}}
    % \WithSuffix\def\PiOneOne(#1){\PiOneN[#1]{1}}
    \newcommand*{\PiOneTwo}{\PiOneN{2}}
    \WithSuffix\def\PiOneTwo[#1]{\PiOneN[#1]{2}}
    % \WithSuffix\def\PiOneTwo(#1){\PiOneN[#1]{2}}
    \newcommand*{\PiOneThree}{\PiOneN{3}}
    \WithSuffix\def\PiOneThree[#1]{\PiOneN[#1]{3}}
    % \WithSuffix\def\PiOneThree(#1){\PiOneN[#1]{3}}



    %%%ABREVIATIONS
    \newcommand*{\Deltaz}{\deltaN{0}}
    \newcommand*{\Deltazn}{\DeltaZeroN}
    \newcommand*{\Deltazii}{\DeltaZeroTwo}
    \newcommand*{\Deltazi}{\DeltaZeroOne}
    \newcommand*{\Deltazz}{\DeltaZeroZero}
    \newcommand*{\Deltaziii}{\DeltaZeroThree}
    \newcommand*{\DeltaIn}{\DeltaOneN}
    \newcommand*{\DeltaIi}{\DeltaOneOne}
    \newcommand*{\Deltaoneone}{\DeltaOneOne}
    \newcommand*{\DeltaIii}{\DeltaOneTwo}
    \newcommand*{\DeltaIiii}{\DeltaOneThree}
    \newcommand*{\Sigmazn}{\SigmaZeroN}
    \newcommand*{\Sigmazi}{\SigmaZeroOne}
    \newcommand*{\Sigmazii}{\SigmaZeroTwo}
    \newcommand*{\Sigmaziii}{\SigmaZeroThree}
    \newcommand*{\SigmaIn}{\SigmaOneN}
    \newcommand*{\SigmaIi}{\SigmaOneOne}
    \newcommand*{\SigmaIii}{\SigmaOneTwo}
    \newcommand*{\SigmaIiii}{\SigmaOneThree}
    \newcommand*{\Pizn}{\PiZeroN}
    \newcommand*{\Pizi}{\PiZeroOne}
    \newcommand*{\Pizii}{\PiZeroTwo}
    \newcommand*{\Piziii}{\PiZeroThree}
    \newcommand*{\PiIn}{\PiOneN}
    \newcommand*{\PiIi}{\PiOneOne}
    \newcommand*{\PiIii}{\PiOneTwo}
    \newcommand*{\PiIiii}{\PiOneThree}






\fi





\ifrecthy@nospaces
    \@packinfo{No spaces loading.}
\else
    \@packinfo{Spaces loading.}

    \providecommand*{\bstrs}{2^{<\omega}}
    \providecommand*{\wstrs}{\omega^{<\omega}}

    \providecommand*{\cantor}{2^{\omega}}
    % \WithSuffix\def\cantor*{\left(2\union \diverge \right)^{\omega}}
    \providecommand*{\baire}{\omega^{\omega}}
    % \WithSuffix\def\baire*{\left(\omega \union \diverge \right)^{\omega}}
    \providecommand*{\Baire}{\mathcal{N}}
\fi







\ifrecthy@notrees
    \@packinfo{No trees loading.}
\else
    \@packinfo{Trees loading.}
    \newcommand*{\CBderiv}[2][1]{#2^{\langle#1\rangle}}
    \newcommand*{\pruneTree}[1]{\CBderiv[\infty]{#1}}
    \newcommand*{\hgt}[1]{\lVert #1 \rVert}
\fi







%re set relations
\ifrecthy@nosetrels
    \@packinfo{No setrels loading.}
\else
    \@packinfo{setrels loading.}
    \AtBeginDocument{
        \providecommand*{\nsupset}{\not\supset@recthy@orig}
        \providecommand*{\nsubset}{\not\subset@recthy@orig}
        \let\nsubset@recthy@orig=\nsubset
        \let\nsupset@recthy@orig=\nsupset
        \let\subset@recthy@orig=\subset
        \let\supset@recthy@orig=\supset
        \let\subseteq@recthy@orig=\subseteq
        \let\supseteq@recthy@orig=\supseteq
        \let\subsetneq@recthy@orig=\subsetneq
        \let\supsetneq@recthy@orig=\supsetneq
    }
        \newcommand*{\subset@recthy}{\subset@recthy@orig}
        \newcommand*{\supset@recthy}{\supset@recthy@orig}
        \newcommand*{\subseteq@recthy}{\subseteq@recthy@orig}
        \newcommand*{\supseteq@recthy}{\supseteq@recthy@orig}
        \newcommand*{\nsubset@recthy}{\nsubset@recthy@orig}
        \newcommand*{\nsupset@recthy}{\nsupset@recthy@orig}
        \newcommand*{\nsubseteq@recthy}{\nsubseteq@recthy@orig}
        \newcommand*{\nsupseteq@recthy}{\nsupseteq@recthy@orig}
        \WithSuffix\def\subset@recthy*{\subset@recthy@orig^{*}}
        \WithSuffix\def\supset@recthy*{\supset@recthy@orig^{*}}
        \WithSuffix\def\subseteq@recthy*{\subseteq@recthy@orig^{*}}
        \WithSuffix\def\supseteq@recthy*{\supseteq@recthy@orig^{*}}
        \WithSuffix\def\subsetneq@recthy*{\subsetneq@recthy@orig^{*}}
        \WithSuffix\def\supsetneq@recthy*{\supsetneq@recthy@orig^{*}}
        \WithSuffix\def\nsubset@recthy*{\nsubset^{*}}
        \WithSuffix\def\nsupset@recthy*{\nsupset^{*}}
        \WithSuffix\def\nsubseteq@recthy*{\nsubseteq^{*}}
        \WithSuffix\def\nsupseteq@recthy*{\nsupseteq^{*}}
        \AtBeginDocument{
          \let\subset=\subset@recthy
          \let\supset=\supset@recthy
          \let\subseteq=\subseteq@recthy
          \let\supseteq=\supseteq@recthy
          \let\subsetneq=\subsetneq@recthy
          \let\supsetneq=\supsetneq@recthy
          \let\nsubset=\nsubset@recthy
          \let\nsupset=\nsupset@recthy
          \let\nsubseteq=\nsubseteq@recthy
          \let\nsupseteq=\nsupseteq@recthy
        }

        \newcommand*{\eq}{=}
        \WithSuffix\def\eq*{=^{*}}
        \newcommand*{\eqae}{=^{*}}
            \newcommand*{\neqae}{\neq^{*}}
        \WithSuffix\def\neq*{\neq^{*}}

        \newcommand*{\infsubset}{\subset_{\infty}}
        \WithSuffix\def\infsubset*{\infsubset^{*}}
        \let\subsetnaeq=\infsubset
        \newcommand*{\infsupset}{\supset_{\infty}}
        \WithSuffix\def\infsupset*{\infsupset^{*}}
        \let\supsetnaeq=\infsupset
        \newcommand*{\majsubset}{\subset_{m}}
        \newcommand*{\majsupset}{\supset_{m}}
\fi












%%% ORdinal notations
% constructive/recursive ordinals

\ifrecthy@noordinalnotations
    \@packinfo{No ordinal notations loading.}
\else
    \@packinfo{ordinal notations loading.}

\providecommand*{\wck}{\ensuremath{\omega^{\mathrm{CK}}_1}}
  % \newcommand*{\ordtype}[1]{\ensuremath{\@recthy@abs{#1}_{\kleeneO}}}
    \newcommand*{\ordzero}{\ensuremath{0}}
  \def\kleeneOSYM{\mathcal{O}}
   \NewDocumentCommand{\kleeneO}{sD(){}oD(){}}{{
       \kleeneOSYM^{#2#4}_{%
            \IfNoValueTF{#1}{%
                \IfNoValueTF{#3}{}{%
                    \abs{#3}}% 
                }%
                {%
                    1%
                    \IfNoValueTF{#3}{}{%
                        , \abs{#3}
                    }%
                }%
        }%
    }}
    % \newcommand*{\kleeneO}{\let\@PMG@parenarg\@PMG@undefined\let\@PMG@braketarg\@PMG@undefined\@kleeneObody}
    % \newcommand*{\@kleeneObody}{\kleeneOSYM\ifdefined\@PMG@parenarg
    % ^{\@PMG@parenarg}%
    % \fi%
    % \ifdefined\@PMG@braketarg
    % _{\abs{\@PMG@braketarg}}%
    % \fi}
    % \WithSuffix\def\@kleeneObody(#1){\def\@PMG@parenarg{#1}\@kleeneObody}
    % \WithSuffix\def\@kleeneObody[#1]{\def\@PMG@braketarg{#1}\@kleeneObody}
    % \WithSuffix\def\@kleeneObody*{\def\kleeneOSYM{\mathcal{O}}\@kleeneObody}

% \newcommand*{\kleeneObelow}[2][]{\kleeneOSYM_{<#2\ifthenelse{\isempty{#1}}{}{, #1}}}
% \let\kleeneOless\kleeneObelow

    \newcommand*{\kleeneOuniq}{\kleeneO*}
    \let\ordNotations=\kleeneO
    \let\uniqOrdNotations=\kleeneOuniq
    \newcommand*{\kleeneNum}[1]{{#1}_{\kleeneO}}
    \newcommand*{\kleeneZero}{\kleeneNum{0}}
    \newcommand*{\kleeneOne}{\kleeneNum{1}}
    \NewDocumentCommand{\kleeneless}{o}{<_{\kleeneO\IfNoValueTF{#1}{}{,#1}}}
    \let\kleenel=\kleeneless
    \NewDocumentCommand{\kleeneleq}{o}{\leq_{\kleeneO\IfNoValueTF{#1}{}{,#1}}}
    \NewDocumentCommand{\kleenenless}{o}{\nless_{\kleeneO\IfNoValueTF{#1}{}{,#1}}}
    \let\kleenenl=\kleenenless
    \NewDocumentCommand{\kleenenleq}{o}{\nleq_{\kleeneO\IfNoValueTF{#1}{}{,#1}}}
    \NewDocumentCommand{\kleenegtr}{o}{>_{\kleeneO\IfNoValueTF{#1}{}{,#1}}}
    \let\kleeneg=\kleenegtr
    \NewDocumentCommand{\kleenegeq}{o}{\geq_{\kleeneO\IfNoValueTF{#1}{}{,#1}}}
    \NewDocumentCommand{\kleenengtr}{o}{\ngtr_{\kleeneO\IfNoValueTF{#1}{}{,#1}}}
    \let\kleeneng=\kleenengtr
    \NewDocumentCommand{\kleenengeq}{o}{\ngeq_{\kleeneO\IfNoValueTF{#1}{}{,#1}}}
    \newcommand*{\kleenePlus}{\mathbin{+_{\kleeneO}}}
    \newcommand*{\kleeneMul}{\cdot_{\kleeneO}}
    \newcommand*{\kleenehgt}[1]{\lVert #1\rVert_{\kleeneO}}
    \let\kleeneHgt=\kleenehgt
    % \let\hgtO=\kleenehgt
    \newcommand*{\kleenelim}[2]{{#1}_{[#2]}}
    \newcommand*{\kleenepred}[1]{{#1}^{-}}

\fi





% \newcommand*{\axiom}{2}{#1}
%alpha REA stuff
% \newcommand*{\alphaREA}[1][\alpha]{\ensuremath{ #1 \text{-REA}} }


\newcommand*{\@PMG@VdashStar}{{\Vdash\smash{\mkern-12mu\raise4\p@\hbox{\text{\tiny*}}}\mkern2mu}}

% Forcing
\ifrecthy@noforcing
    \@packinfo{No forcing loading.}
\else
    \@packinfo{forcing loading.}
    \providecommand*{\forces}{\def\@PMG@frcparenarg{}\def\@PMG@frcbraketarg{}\def\@PMG@curfrcsym{\Vdash}\@forcesBody}
    \newcommand*{\@forcesBody}{\mathrel{{\@PMG@curfrcsym}^{\@PMG@frcparenarg}_{\@PMG@frcbraketarg}}}
    \WithSuffix\def\@forcesBody(#1){\def\@PMG@frcparenarg{#1}\@forcesBody}
    \WithSuffix\def\@forcesBody[#1]{\def\@PMG@frcbraketarg{#1}\@forcesBody}
    \WithSuffix\def\@forcesBody*{\let\@PMG@curfrcsym=\@PMG@VdashStar\@forcesBody}

    % \long\def\forces@[#1][#2]{\Vdash^{#2}_{#1}}
    % \newcommand{\forces}{\optparams{\forces@}{[][]}%
    % }
    \providecommand*{\frc}{\forces}
\fi




%%%ABREVIATIONS


%Degree Classes
\ifrecthy@nodegclasses
    \@packinfo{No degree classes loading.}
\else
    \@packinfo{Degree classes loading.}
    \newcommand{\Low}[1][]{\ensuremath{ \text{Low}^{#1} }}
    \newcommand{\LowN}[2][]{\ensuremath{ \text{Low}^{#1}_{#2} }}
    \newcommand{\lowN}[2][]{\ensuremath{ \mathcal{L}^{#1}_{#2} }}
\fi



\ifrecthy@nocomputations
    \@packinfo{No computations loading.}
\else
%%%%%%
    \newcommand*{\murec}[2]{\ensuremath{\mu#1\left( #2 \right) }}

    %\newcommand*{\brecfSYM}{\phi}

    % \newcommand*{\recf}[3][]{\@recthy@recfSYM^{#1}_{#2}\ifthenelse{\isempty{#3}}{}{\left(#3\right)}} %If give an empty arg just print function no parens
    %\newcommand*{\brecf}[3][]{\brecfSYM^{#1}_{#2}\ifthenelse{\isempty{#3}}{}{\left(#3\right)}}
    %\recfnl[d]{i}{Y}{x} gives i-th d-recursive functional operating on set Y and integer x

    \NewDocumentCommand{\recfnl@improved}{oD(){}m!G{#2}!g!d()}{{
      \def\temp@pmg@arg{}
      \IfValueTF{#5}{\ifthenelse{\isempty{#5}}{}{\def\temp@pmg@arg{; #5}}}{\IfValueTF{#6}{\ifthenelse{\isempty{#6}}{}{\def\temp@pmg@arg{; #6}}}}
        \@recthy@recfnlSYM_{#3\IfValueT{#1}{, #1}}\ifthenelse{\isempty{#4}}{}{\!\left(#4\temp@pmg@arg \right)}
    }}

    \NewDocumentCommand{\recfnl@threeSix}{omm!g!d()}{{
      \def\temp@pmg@arg{}
      \IfValueTF{#4}{\ifthenelse{\isempty{#4}}{}{\def\temp@pmg@arg{; #4}}}{\IfValueTF{#5}{\ifthenelse{\isempty{#5}}{}{\def\temp@pmg@arg{; #5}}}}
        \@recthy@recfnlSYM_{#2\IfValueT{#1}{, #1}}\ifthenelse{\isempty{#3}}{}{\!\left(#3\temp@pmg@arg \right)}
    }}
    \newcommand*{\recfnl@threeFive}[4][]{\@recthy@recfnlSYM_{#2\ifthenelse{\isempty{#1}}{}{,{#1}} }%
        \ifthenelse{\isempty{#4}}%
        {\ifthenelse{\isempty{#3}}%
            {}% No args then print no parents
            {(#3)}
        }%If no oracle we just refer to the functional without any arguments
        {\ifthenelse{\isempty{#3}}{(#4)}{(#3;#4)}%
    }}
    \let\recfnl=\recfnl@improved
    \appto{\recthy@rollback@threeSix}{
      \let\recfnl=\recfnl@threeFive
    }
    \appto{\recthy@rollback@threeSeven}{
      \let\recfnl=\recfnl@threeSix
    }
    \NewDocumentCommand{\recf}{od()md()}{
        \@recthy@recfSYM_{#3\IfNoValueTF{#1}{}{\ifthenelse{\isempty{#3}}{}{, }{#1}}}\IfNoValueTF{#2}{}{^{#2}}\IfNoValueTF{#4}{}{(#4)}
    }


    % \newcommand*{\thetafnl}[4][]{\Theta_{#2\ifthenelse{\isempty{#1}}{}{,#1} }%
    %     \ifthenelse{\isempty{#4}}%
    %     {\ifthenelse{\isempty{#3}}%
    %         {}% No args then print no parents
    %         {(#3)}
    %     }%If no oracle we just refer to the functional without any arguments
    %     {\ifthenelse{\isempty{#3}}{(#4)}{(#3;#4)}%
    % }}

    % \newcommand*{\psifnl}[4][]{\Psi_{#2\ifthenelse{\isempty{#1}}{}{,#1} }%
    %     \ifthenelse{\isempty{#4}}%
    %     {\ifthenelse{\isempty{#3}}%
    %         {}% No args then print no parents
    %         {(#3)}
    %     }%If no oracle we just refer to the functional without any arguments
    %     {\ifthenelse{\isempty{#3}}{(#4)}{(#3;#4)}%
    % }}

    % \newcommand*{\phifnl}[4][]{\Phi_{#2\ifthenelse{\isempty{#1}}{}{,#1} }%
    %     \ifthenelse{\isempty{#4}}%
    %     {\ifthenelse{\isempty{#3}}%
    %         {}% No args then print no parents
    %         {(#3)}
    %     }%If no oracle we just refer to the functional without any arguments
    %     {\ifthenelse{\isempty{#3}}{(#4)}{(#3;#4)}%
    % }}

    % \newcommand*{\upsilonfnl}[4][]{\Upsilon_{#2\ifthenelse{\isempty{#1}}{}{,#1} }%
    %     \ifthenelse{\isempty{#4}}%
    %     {\ifthenelse{\isempty{#3}}%
    %         {}% No args then print no parents
    %         {(#3)}
    %     }%If no oracle we just refer to the functional without any arguments
    %     {\ifthenelse{\isempty{#3}}{(#4)}{(#3;#4)}%
    % }}


    \newcommand*{\cequiv}{\mathrel{\simeq}}
    \newcommand*{\ncequiv}{\mathrel{\not\simeq}}
    \NewDocumentCommand{\conv}{O{}}{\mathpunct{\downarrow}_{#1}}
    \let\converge=\conv
    %\newcommand*{\convs}[1]{\ensuremath{\conv_{#1}}}
%   \NewDocumentCommand{\nconv}{o}{\mathpunct{\mkern-4mu{\arrownot\mkern 4mu \downarrow}_{#1}}}
    \NewDocumentCommand{\nconv}{O{}}{\mathpunct{\mkern-4mu{\smallsetminus\mkern 4mu \downarrow}_{#1}}}
    \newcommand*{\diverge}{\mathpunct{\uparrow}}
    \newcommand*{\use}[1]{\mathop{\@recthy@useSYM}\left[#1\right]}
% \newcommand*{\use}[1]{\ltx@ifpackageloaded{unicode-math}{\symbffrak{u}}{\mathfrak{u}}\left(#1\right)}
    % \NewDocumentCommand{\REset}{D(){}omd()}{
    % W_{#2\IfNoValueTF{#1}{}{, #1}}\IfNoValueTF{#3}{}{^{#3}}
    % }

    \newcommand*{\REset}{\let\@PMG@parenarg\@PMG@undefined\let\@PMG@braketarg\@PMG@undefined\@REsetbody}
    \newcommand*{\@REsetbody}[1]{W\ifdefined\@PMG@parenarg
    ^{\@PMG@parenarg}%
    \fi%
    \ifdefined\@PMG@braketarg
    _{#1, {\@PMG@braketarg}}%
    \else
    _{#1}
    \fi}
    \WithSuffix\def\@REsetbody(#1){\def\@PMG@parenarg{#1}\@REsetbody}
    \WithSuffix\def\@REsetbody[#1]{\def\@PMG@braketarg{#1}\@REsetbody}
    \NewDocumentCommand{\Hop}{oD(){}mD(){}}{\mathcal{H}_{#3\IfValueTF{#1}{, #1}{}}\left(\ifthenelse{\isempty{#2#4}}{\eset}{#2#4} \right)}
    \NewDocumentCommand{\REAop}{st+d()od()mm}{%
      {\IfBooleanTF{#1}%
        {\widehat{\mathcal{J}}}%
        {\mathcal{J}}%
      }^{%
          \IfBooleanTF{#2}{\vphantom{x}^\dagger}{}%
          #7}_{#6\IfValueTF{#4}{, #4}{}}%
          \IfValueTF{#3}{\left(#3\right)}{\IfValueTF{#5}{\left(#5\right)}{}}%
    }
      \NewDocumentCommand{\oneREAop}{st+d()od()m}{{\IfBooleanTF{#1}{\mathcal{J}}{\widehat{\mathcal{J}}}}^{\IfBooleanTF{#1}{\vphantom{x}^\dagger}{}}_{#6\IfValueTF{#4}{, #4}{}} \IfValueTF{#3}{\left(#3\right)}{\IfValueTF{#5}{\left(#5\right)}{}}}
    \let\iREAop=\oneREAop

    \newcommand*{\entersat}[1]{\mathbin{\searrow_{#1}}}
    

    % \newcommand*{\alphaREAop}[2][]{\mathcal{J}^{#2}_{#1}}
    \appto{\recthy@rollback@threeTwo}{
        \let\reaop=\REAop
        \let\alphaREAop=\REAop
        \let\aREAop=\alphaREAop
    }


    

    % \newcommand*{\oneREAop}[1]{J_{#1}}
    % \let\iREAop=\oneREAop
    % \WithSuffix\def\reaop*#1{\oneREAop{#1}}
    %converges

\fi



\ifrecthy@nodegrees
    \@packinfo{No degrees loading.}
\else


    %%%%%%Turing Degrees
    \def\firsttokof@pmg#1{\first@pmg@rdmxs(#1)}
    \def\first@pmg@rdmxs(#1#2){#1}

    \providecommand*{\Tdegvar}[1]{\symbf{#1}}
    \providecommand*{\Tdegof}[1]{\utilde{#1}}
    \providecommand*{\Tdeg}[1]{\ifthenelse{
    % \equal{\detokenize{\Tzero}}{\futurelet\next@pmg\discard#1\@nil\expandafter\detokenize{\next@pmg}}
                                        \equal{\detokenize{\Tzero}}{\detokenize{#1}} \OR  
                                        \equal{\detokenize{\Tzero}}{\detokenize{#1}} \OR 
                                        \equal{\detokenize{\zeroj}}{\detokenize{#1}} \OR 
                                        \equal{\detokenize{\zerojj}}{\detokenize{#1}} \OR 
                                        \equal{\detokenize{\zerojjj}}{\detokenize{#1}} \OR 
                                        \equal{\string \zeron }{\expandafter\string \first@pmg@rdmxs(#1)}  
                                       }{\Tdegof{#1}}{\Tdegvar{#1}}}

    \providecommand*{\Tjump}[1]{#1'}
    \let\jump=\Tjump
    \providecommand*{\jumpn}[2]{#1^{(#2)}}
    \providecommand*{\jjump}[1]{#1''}
    % \newcommand*{\Tzerosym}{\MVZero}
    \newcommand*{\Tzerosym}{\mathbb{0}}
    \providecommand*{\Tzero}{{\Tzerosym}}
    \providecommand*{\zeron}[1]{{\Tzerosym}^{(#1)}}
    \providecommand*{\zeroj}{\jump{{\Tzerosym}}}
    \let\Tzeroj=\zeroj
    \providecommand*{\zerojj}{\jump{\jump{{\Tzerosym}}}}
    \providecommand*{\zerojjj}{\jump{\jump{\jump{{\Tzerosym}}}}}
    \let\Tzerojj=\zerojj
    \let\Tzerojjj=\zerojjj

    %%Misc Recursion theory stuff
    %turing equivalent
    \newcommand*{\@recthy@TSYM}{\symbf{T}}
    \providecommand*{\Tequiv}{\mathrel{\equiv_{\@recthy@TSYM}}}
    \providecommand*{\Teq}{\Tequiv}
    \providecommand*{\nTequiv}{\mathrel{\ncong_{\@recthy@TSYM}}}
    \providecommand*{\nTeq}{\nTequiv}
    %turing below

    \providecommand*{\Tlneq}{\lneq_{\@recthy@TSYM}}
    \providecommand*{\Tleq}{\leq_{\@recthy@TSYM}}
    \providecommand*{\Tgneq}{\gneq_{\@recthy@TSYM}}
    \providecommand*{\Tgeq}{\geq_{\@recthy@TSYM}}
    \providecommand*{\Tgtr}{>_{\@recthy@TSYM}}
    \providecommand*{\Tless}{<_{\@recthy@TSYM}}
    \providecommand*{\nTleq}{\nleq_{\@recthy@TSYM}}
    \providecommand*{\nTgeq}{\ngeq_{\@recthy@TSYM}}

    \providecommand*{\Tdegjoin}{\mathbin{\vee_{\@recthy@TSYM}}}
    \providecommand*{\Tdegmeet}{\mathbin{\wedge_{\@recthy@TSYM}}}
    \providecommand*{\Tmeet}{\Tdegmeet}
    \providecommand*{\Tsetjoin}{\mathbin{\oplus}}
    \providecommand*{\TsetJoin}{\mathop{\bigoplus}}
    \providecommand*{\Tplus}{\Tsetjoin}
    \providecommand*{\TPlus}{\TsetJoin}
    \providecommand*{\Tjoin}{\Tsetjoin}
    \providecommand*{\TJoin}{\TsetJoin}
    \providecommand*{\Tincompat}{\incompat@recthy[\@recthy@TSYM]}
    \providecommand*{\Tincomp}{\Tincompat}
    \providecommand*{\Tcompat}{\compat@recthy[\@recthy@TSYM]}
    \providecommand*{\nTincompat}{\Tcompat}
    \providecommand*{\nTincomp}{\Tcompat}



    \newcommand*{\@recthy@ttSYM}{\ensuremath{\mathtt{tt}}}
    \newcommand*{\ttSYM}{\@recthy@ttSYM}
    \newcommand*{\ttlneq}{\lneq_{\@recthy@ttSYM}}
    \newcommand*{\ttleq}{\leq_{\@recthy@ttSYM}}
    \newcommand*{\ttgneq}{\gneq_{\@recthy@ttSYM}}
    \newcommand*{\ttgeq}{\geq_{\@recthy@ttSYM}}
    \newcommand*{\ttgtr}{>_{\@recthy@ttSYM}}
    \newcommand*{\ttless}{<_{\@recthy@ttSYM}}
    \newcommand*{\ttnleq}{\nleq_{\@recthy@ttSYM}}
    \newcommand*{\ttngeq}{\ngeq_{\@recthy@ttSYM}}
\fi


\ifrecthy@stepsenv

  \newlist{pmg@steps}{enumerate}{4}
  \setlist[pmg@steps,1]{label={Step \arabic*}, ref={\arabic*}, labelsep=1em}
  \setlist[pmg@steps,2]{label={Step \arabic{pmg@stepsi}\alph*}, ref={\arabic{pmg@stepsi}\alph*}}
  \setlist[pmg@steps,3]{label={Step \arabic{pmg@stepsi}\alph{pmg@stepsii}.\Roman*}, ref={\arabic{pmg@stepsi}\alph{pmg@stepsii}.\Roman*}}
  \setlist[pmg@steps,4]{label={Step \arabic{pmg@stepsi}\alph{pmg@stepsii}.\Roman{pmg@stepsiii}\Alph*}, ref={\arabic{pmg@stepsi}\alph{pmg@stepsii}.\Roman{pmg@stepsiii}\Alph*}}
  \crefname{pmg@stepsi}{step}{steps}
  \Crefname{pmg@stepsi}{Step}{Steps}
  \crefname{pmg@stepsii}{step}{steps}
  \Crefname{pmg@stepsii}{Step}{Steps}
  \crefname{pmg@stepsiii}{step}{steps}
  \Crefname{pmg@stepsiii}{Step}{Steps}
  \crefname{pmg@stepsiiii}{step}{steps}
  \Crefname{pmg@stepsiiii}{Step}{Steps}
  \NewDocumentEnvironment{steps}{o}{
      \begingroup
      \ProvideDocumentCommand{\step}{o}{\IfValueTF{##1}{\item\textbf{##1}:}{\item}}
      \IfValueTF{#1}{\begin{pmg@steps}[#1]}{\begin{pmg@steps}}%
  }{%
      \end{pmg@steps}%
      \endgroup\ignorespacesafterend%
      }


  % \ExplSyntaxOn
  % \cs_new:cpn {steps*} {\steps*}
  % \cs_new_eq:cN {endsteps*} \endsteps
  % \ExplSyntaxOff
\else
    \@packinfo{No steps}
\fi

